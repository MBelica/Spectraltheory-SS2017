\documentclass[12pt]{extreport} % Schriftgröße: 8pt, 9pt, 10pt, 11pt, 12pt, 14pt, 17pt oder 20pt

%% Packages
\usepackage{scrextend}
\usepackage{amssymb}
\usepackage{amsthm}
\usepackage{booktabs}
\usepackage{chngcntr}
\usepackage{cmap}
\usepackage{color}
\usepackage{csquotes}
\usepackage{enumitem}
\usepackage{float}
\usepackage{bbding}
\usepackage{hyperref}
\usepackage{ulem}
\usepackage{lmodern}
\usepackage{makeidx}
\usepackage{mathtools}
\usepackage{xpatch}
\usepackage{pgfplots}
\pgfplotsset{compat=1.7}
\usetikzlibrary{calc}	
\usetikzlibrary{matrix}	

% Language Setup (English)
\usepackage[utf8]{inputenc} 
\usepackage[T1]{fontenc} 
\usepackage[english]{babel}

% Options
\makeatletter%%  
  % Linkfarbe, {0,0.35,0.35} für Türkis, {0,0,0} für Schwarz, {1,0,0} für Rot, {0,0,0.85} für Blau
  \definecolor{linkcolor}{rgb}{0,0.35,0.35}
  % Zeilenabstand für bessere Leserlichkeit
  \def\mystretch{1.2} 
  % Publisher definieren
  \newcommand\publishers[1]{\newcommand\@publishers{#1}} 
  % Enumerate im 1. Level: \alph für a), b), ...
  \renewcommand{\labelenumi}{\alph{enumi})} 
  % Enumerate im 2. Level: \roman für (i), (ii), ...
  \renewcommand{\labelenumii}{(\roman{enumii})}
  % Zeileneinrückung am Anfang des Absatzes
  \setlength{\parindent}{0pt} 
  % Für das Proof-Environment: 'Beweis:' anstatt 'Beweis.'
  \xpatchcmd{\proof}{\@addpunct{.}}{\@addpunct{:}}{}{} 
  % Nummerierung der Bilder, z.B.: Abbildung 4.1
  \@ifundefined{thechapter}{}{\def\thefigure{\thechapter.\arabic{figure}}} 
  % Chapter-Nummerierung beginnen bei (0):
  \setcounter{chapter}{0}
  % Chapter-Nummerierung
  \renewcommand\thechapter{\Roman{chapter}}
\makeatother%

% Meta Setup 
\title{Spectraltheory - Exercise}
\author{Prof. Dr. Tobias Lamm}
\date{Sommersemester 2017}
\publishers{Karlsruher Institut für Technologie}

%% Math. Definitiones
\newcommand{\C}{\mathbb{C}}
\newcommand{\N}{\mathbb{N}}
\newcommand{\Q}{\mathbb{Q}}
\newcommand{\R}{\mathbb{R}}
\newcommand{\Z}{\mathbb{Z}}
\newcommand{\DO}[1]{\mathcal{D}\left( {#1} \right)}
\newcommand{\RO}[1]{\mathcal{R}\left( {#1} \right)}

\newtheoremstyle{named}{}{}{\normalfont}{}{\bfseries}{:}{0.25em}{#2 \thmnote{#3}}
\newtheoremstyle{nnamed}{}{}{\normalfont}{}{\bfseries}{:}{0.25em}{\thmnote{#3}}
\newtheoremstyle{itshape}{}{}{\itshape}{}{\bfseries}{:}{ }{}
\newtheoremstyle{normal}{}{}{\normalfont}{}{\bfseries}{:}{ }{}
\renewcommand*{\qed}{\hfill\ensuremath{\square}}

\theoremstyle{named}
\newtheorem{unnamedtheorem}{Theorem} \counterwithin{unnamedtheorem}{chapter}
\theoremstyle{nnamed}
\newtheorem*{unnamedtheorem*}{Theorem} 

\theoremstyle{itshape}
\newtheorem{definition}{Definition}  \counterwithin{definition}{chapter}
\newtheorem{theorem}{Theorem}  \counterwithin{theorem}{chapter}
\newtheorem{lemma}{Lemma}  \counterwithin{lemma}{chapter}

\theoremstyle{normal}
\newtheorem*{recall}{Recall}
\newtheorem*{example}{Example}
\newtheorem*{remark}{Remark}
\newtheorem*{satz}{Satz}
\newtheorem*{bemerkung}{Bemerkung}

%% Template
\makeatletter%
\DeclareUnicodeCharacter{00A0}{ } \pgfplotsset{compat=1.7} \hypersetup{colorlinks,breaklinks, urlcolor=linkcolor, linkcolor=linkcolor, pdftitle=\@title, pdfauthor=\@author, pdfsubject=\@title, pdfcreator=\@publishers}\DeclareOption*{\PassOptionsToClass{\CurrentOption}{report}} \ProcessOptions \def\baselinestretch{\mystretch} \setlength{\oddsidemargin}{0.125in} \setlength{\evensidemargin}{0.125in} \setlength{\topmargin}{0.5in} \setlength{\textwidth}{6.25in} \setlength{\textheight}{8in} \addtolength{\topmargin}{-\headheight} \addtolength{\topmargin}{-\headsep} \def\pulldownheader{ \addtolength{\topmargin}{\headheight} \addtolength{\topmargin}{\headsep} \addtolength{\textheight}{-\headheight} \addtolength{\textheight}{-\headsep} } \def\pullupfooter{ \addtolength{\textheight}{-\footskip} } \def\ps@headings{\let\@mkboth\markboth \def\@oddfoot{} \def\@evenfoot{} \def\@oddhead{\hbox {}\sl \rightmark \hfil \rm\thepage} \def\chaptermark##1{\markright {\uppercase{\ifnum \c@secnumdepth >\m@ne \@chapapp\ \thechapter. \ \fi ##1}}} \pulldownheader } \def\ps@myheadings{\let\@mkboth\@gobbletwo \def\@oddfoot{} \def\@evenfoot{} \def\sectionmark##1{} \def\subsectionmark##1{}  \def\@evenhead{\rm \thepage\hfil\sl\leftmark\hbox {}} \def\@oddhead{\hbox{}\sl\rightmark \hfil \rm\thepage} \pulldownheader }	\def\chapter{\cleardoublepage  \thispagestyle{plain} \global\@topnum\z@ \@afterindentfalse \secdef\@chapter\@schapter} \def\@makeschapterhead#1{ {\parindent \z@ \raggedright \normalfont \interlinepenalty\@M \Huge \bfseries  #1\par\nobreak \vskip 40\p@ }} \newcommand{\indexsection}{chapter} \patchcmd{\@makechapterhead}{\vspace*{50\p@}}{}{}{}\def\Xint#1{\mathchoice
    {\XXint\displaystyle\textstyle{#1}} {\XXint\textstyle\scriptstyle{#1}} {\XXint\scriptstyle\scriptscriptstyle{#1}} {\XXint\scriptscriptstyle\scriptscriptstyle{#1}} \!\int} \def\XXint#1#2#3{{\setbox0=\hbox{$#1{#2#3}{\int}$} \vcenter{\hbox{$#2#3$}}\kern-.5\wd0}} \def\dashint{\Xint-} \def\Yint#1{\mathchoice {\YYint\displaystyle\textstyle{#1}} {\YYYint\textstyle\scriptscriptstyle{#1}} {}{} \!\int} \def\YYint#1#2#3{{\setbox0=\hbox{$#1{#2#3}{\int}$} \lower1ex\hbox{$#2#3$}\kern-.46\wd0}} \def\YYYint#1#2#3{{\setbox0=\hbox{$#1{#2#3}{\int}$}  \lower0.35ex\hbox{$#2#3$}\kern-.48\wd0}} \def\lowdashint{\Yint-} \def\Zint#1{\mathchoice {\ZZint\displaystyle\textstyle{#1}}{\ZZZint\textstyle\scriptscriptstyle{#1}} {}{} \!\int} \def\ZZint#1#2#3{{\setbox0=\hbox{$#1{#2#3}{\int}$}\raise1.15ex\hbox{$#2#3$}\kern-.57\wd0}} \def\ZZZint#1#2#3{{\setbox0=\hbox{$#1{#2#3}{\int}$} \raise0.85ex\hbox{$#2#3$}\kern-.53\wd0}} \def\highdashint{\Zint-} \DeclareRobustCommand*{\onlyattoc}[1]{} \newcommand*{\activateonlyattoc}{ \DeclareRobustCommand*{\onlyattoc}[1]{##1} } \AtBeginDocument{\addtocontents{toc} {\protect\activateonlyattoc}} 
	% Titlepage
	\def\maketitle{ \begin{titlepage} 
			~\vspace{3cm} 
		\begin{center} {\Huge \@title} \end{center} 
	 		\vspace*{1cm} 
	 	\begin{center} {\large \@author} \end{center} 
	 	\vspace*{-0.5cm}
	 	\begin{center} \@date \end{center} 
	 		\vspace*{7cm} 
	 	\begin{center} \@publishers \end{center} 
	 		\vfill 
	\end{titlepage} }
\makeatother%

% Create Index
\makeindex 

\begin{document}

\pagenumbering{Alph}
\begin{titlepage}
	\maketitle
	\thispagestyle{empty}
\end{titlepage}

% Lecture Notes - Start 			
\pagenumbering{arabic}
% Lecture Notes - End 
\appendix % todo appendix
% Exercises - Start 

\chapter*{Exercises}

\subsection*{Exercise 1}

\begin{enumerate}
	\item $H$ separable $\Rightarrow \exists (e_n)_{n \in \N} \subseteq H$ orthonormal basis of $H$.
		\begin{proof}
			$H$ separable $\Rightarrow \exists (u_n)_{n \in \N} \subseteq H$: $\overline{ \{ u_n | n \in \N \}} = H$. Define:
			$$ H_n \coloneqq \operatorname{lin} \{ u_1, \dotsc, u_n \}, ~ \text{ for } n \in \N $$
			$H_n \subseteq H$ closed subspace of $H$ as $\dim H_n \leq n < \infty$.
			By Projektionssatz there exists an orthogonal projection $P_n$ on $H_n$. Set
			$$ g_n \coloneqq u_n - P_{n-1} un \in H_{n} \cap H_{n-1}^{\perp}, N \coloneqq \{ n \in \N | g_n \neq 0 \} $$
			Now we define 
			$$ e_n \coloneqq \begin{cases} \frac{g_{n}}{\| g_n \}}, & n \in N \\ 0, & n \notin N \end{cases} $$
			$\Rightarrow e_n \in H_n \cap H_{n-1}^{\perp}$, $\operatorname{lin}\{. e_1, \dotsc, e_n \} = H_n \Rightarrow (e_n)_{n \in \N}$ is an orthonormal basis of $H$.
		\end{proof}
	\item $H = L^2(0, 1)$, $\DO(T) = W^{1,2}(0, 1) \eqqcolon H^{1}(0,1)$, $T f = i f'$.
		\begin{proof}
			$T$ abgeschlossen: $(x_n)_{n \in \N} \subseteq \DO{T}$ Cauchyfolge bezüglich $\| \cdot \|_{W^{1,2}}$. $\xRightarrow[\text{vollständig}]{L^2} \exists x, y \in L^2(0,1): x_n \rightarrow x, x_n' \rightarrow y \text{ in } L^{2}(0,1)$
			$$ \Rightarrow \int_0^1 x \varphi' dt = \lim_{n \rightarrow \infty} \int_0^1 x_n \varphi' dt = - \lim_{n \rightarrow \infty} \int_0^1 x_n'' \varphi dt = - \int_0^1 y \varphi dt ~\forall \varphi \in C_c^{\infty}(0,1) $$
			$\Rightarrow x \in \DO{T}$, $x' = y$, $T x = i x' \Rightarrow T$ abgeschlossen. ~\\
			We still have to show that $T$ isn't symmetric:
			$$ \langle  Tx, y \rangle_{L^2} = \int_0^1 i x' \overline{y} dt \overset{P.I.} {=}[ixy]_0^1 - \int_0^1 i x \overline{y}' dt = \underbrace{[ixy]_0^1}_{\neq 0 \text{ i. g.}} + \langle x, Ty \rangle_{L^2},$$
			d.h. $T$ is not symmetrich $\Rightarrow t$ nicht self-adjoint. $T$ isn't halb-beschränkt nach unten;: siehe $(iii)$.
		\end{proof}
	\item $\DO{T} = W_0^{1,2}(0,1)$, $H = L^{2}(0, 1)$, $Tf = if'$. 
		\begin{proof}
			$T$ closed: as in $(ii)$, ~\\
			$T$ symmetrisch:
			$$ \langle Tx, y \rangle_{L^2} = \dotsc = \underbrace{[ixy]_0^1}_{= 0} + \langle x, Ty \rangle_{L^2} = \langle x, Ty \rangle ~ \forall x, y \in \DO{T}$$
			$T$ not self-adjoint:
			$$ \DO{T^*} = \big\{ y\in L^2(0, 1) \colon x \mapsto \langle Tx, y \rangle_{L^2} \text{ continuous on } \DO{T} \big\} $$
			Vermutung: $W^{1,2}(0, 10) \subseteq \DO{T^*}$ (even \enquote{=}).
			Let $x \in \DO{T}, y \in W^{1,2}(0,1)$:
			$$ \langle Tx, y \rangle_{L^2} = \dotsc = \underbrace{[ixy]_0^1}_{= 0} + \langle x, iy' \rangle_{L^2} = \langle x, iy' \rangle_{L^2}  $$
			continuous on $\DO{T}$, i.e. $W^{1,2}(0,1) \subseteq \DO{T^*}$, however $W^{1,2}(0, 1) \not\subseteq W_0^{1,2}(0,1)$, i.e. $\DO{T} \neq \DO{T^*} \Rightarrow T \neq T^*$. ~\\
			$T$ is not halb-beschränkt nach unten: 
			$$ {\color{gray} \textit{ Consider the comment: } \langle Tx, x \rangle_{L^2} = - 2 \int_0^21 \left( \operatorname{Im} x \right)' \operatorname{Re}x dt \overset{\textit{\enquote{?}}}{\geq}  c \langle x, x \rangle_{L^2} } $$
			For $f_0 \in W_0^{1,2}(0, 1)$ with $\langle f_0, f_0 \rangle_{L^2} = 1$, $w \in \R$, $f_w(t) \coloneqq e^{iwt} f_0(t) \Rightarrow \langle f_w, f_w \rangle_{L^2} = 1$
			\begin{align*}
				f'_w(t) & = iw e^{iwt} f_0(t) + e^{iwt} f_0'(t) \\ 
						& = iw f_w(t) + e^{iwt} f_0'(t).
			\end{align*}
			\begin{align*}
				\langle T f_w, f_w \rangle & = \int_0^1 \left( - w f_w(t) + i e^{iwt} f_0'(t) \right) e^{-iwt} \overline{f_0(t)} dt \\ 
					& = \underbrace{\int_0^1 - w |f_0|^2 dt}_{= - w} + \underbrace{ \int_0^1 i f_0'(t) \overline{f_0(t)} dt }_{\langle T f_0, f_0 \rangle_{L^2}} = - w + \underbrace{ \langle T f_0, f_0 \rangle_{L^2}}_{\in \R} \rightarrow \pm \infty 
			\end{align*}
			for $w \rightarrow \pm \infty \Rightarrow T$ ist not halb-beschränkt. ~\bigskip
			In $(iii)$ $T$ has a self-adjoint Erweitunerung $S$:
			$$ \DO{S} = \big\{ x \in W^{1,2}(0,1) \colon x(0) = x(1) \big\}, ~Sf = if' $$
		\end{proof}
\end{enumerate}

\begin{definition}
	Sei $\Omega \subseteq \C$ offen, $r \colon \Omega \rightarrow X$ eine Funktion. Man definiert
	\begin{enumerate}
		\item $r$ ist schwach analytisch $\iff \forall \varphi \in X^* \colon \varphi \circ r$ analytisch auf $\Omega$
		\item $r$ ist analytisch $\iff \frac{d}{dz} r(z_0) \coloneqq r'(z_0) \coloneqq \lim_{z \rightarrow z_0} \left( z - z_0 \right)^{-1} \left[ r(z) - r(z_0) \right]$ existiert in $X$ $\forall z_0 \in \Omega$ 
		\item Kurvenintegrale: Sein $\Gamma \coloneqq \big\{ \gamma(t) \colon t \in [a, b] \big\}$ endlich-stückweise glatte Kurve in $\Omega$, $r$ stetig, dann:
		$$ \int_{\Gamma} r(\lambda) d\lambda \coloneqq \int_a^b r\left( \gamma(t) \right) \cdot \gamma'(t) dt \in X $$
	\end{enumerate}
\end{definition}

\begin{satz}[Lemma von Dunford]
	$r \colon \Omega \rightarrow X$ schwach analytisch $\iff r$ analytisch
\end{satz}

\begin{satz}[Cuachy's Integralsatz und Formel]
	Sei $\Omega \subseteq \C$ offen und konvex, $r \colon \Omega \rightarrow X$ analytisch. Dann gilt:
	\begin{enumerate}
		\item $\gamma \subseteq \Omega$ stückweise glatt und geschlossen $\Rightarrow \int_{\Gamma} r(\lambda) d \lambda  0$.
		\item $\forall \lambda_0 \in \Omega$, $a > 0$ mit $\overline{B(\lambda_0, a)} \subseteq \Omega$:
			$$ r(\lambda) = \frac{1}{2 \pi i} \int_{|\mu - \lambda_0| = a} \frac{1}{\mu - \lambda} r(\mu) d \mu \in X. $$
	\end{enumerate}	
	\begin{proof}
		Sei $x^{*} \in X^*$, dann ist $x^{*} \circ r$ analytisch auf $\Omega$. Nach Integralsatz bzw. -formel aus der Funktionentheorie folgt:
			$$ 0 = \int_{\Gamma} x^{*}\left(r(\lambda) \right) d \lambda = x^{*} \left( \underbrace{\int_{\Gamma} r(\lambda) d\lambda}_{\eqqcolon x_1} \right), $$
			$$ x^*\left( r(\lambda) \right) = \frac{1}{2\pi i} \int_{|\mu - \lambda_0|=a} \frac{1}{\mu - \lambda} x^*\left(r(\mu) \right) d\mu = x^* \left( \frac{1}{2 \pi i} \int_{|\mu - \lambda_0| = a} \frac{1}{\mu - \lambda} r(\mu) d \mu \right) $$
			$$ \iff x^{*} \left( \underbrace{r(\lambda) - \frac{1}{2 \pi i} \int_{\Gamma} \frac{1}{\mu - \lambda} r(\mu) d \mu)}_{\eqqcolon x_2 }  \right)= 0 $$
			$\Rightarrow x^*(x_1) = 0$, $x^*(x_2) = 0 ~\forall x^* \in X^* \xRightarrow[Banach]{Hahn-} x_1 = 0$, $x_2 = 0$.
	\end{proof}
\end{satz}

\subsubsection*{Das Dunford-Kalkül}

\begin{definition}[Kalkül für Polynome]
	Sei $A \in L(X)$, $p \colon \lambda \mapsto \sum_{k = 0}^{n} a_k \lambda^k$ Polynom, $a_k \in \C$ für $k = 0, \dotsc, n$. Dann definiert man:
	$$ p(A) = \sum_{k = 0}^{n} a_k A^k \in L(X). $$
	$\mathcal{P}$ Vektorraum aller Polynome.	
\end{definition}

\begin{satz}[Eigenschaften]
	$p_1, p_2, p \in \mathcal{P}$, $p(\lambda) = \sum_{k=0}^{n} a_k \lambda^k$, $A \in L(X)$, $\alpha, \beta \in \C$. Dann gilt:
	\begin{enumerate}[label=\upshape(\arabic*\upshape)]
		\item Linearität: $\left( \alpha p_1 + \beta p_2 \right) (A) = \alpha p_1(A) + \beta p_2(A)$.
		\item Multiplikativität:: $\left( p_1 \cdot p_2 \right) (A) = p_1(A) p_2(A) = p_2(A) \cdot p_1(A)$.
		\item Beschränktheit: $\|p(A) \|_{L(X)} \leq \sum_{k=0}^{n} |a_k| \| A \|^k_{L(X)}$.
		\item Spektrale Abbildungseigenschaft: $\sigma \left( p(A) \right) = p \left( \sigma(A) \right)$.
	\end{enumerate}
	
	\begin{proof} ~\
		\begin{enumerate}[label=\upshape(\arabic*\upshape)]
			\item - (3): klar.  \setcounter{enumi}{3}
			\item \enquote{$\supseteq$}: Sei $\mu \in \sigma(A)$, dann hat das Polynom $\lambda \mapsto p(\mu)- p(\lambda) \in \mathcal{P}$ eine Nullstelle in $\lambda = \mu$. Somit folgt:
					$$ p(\mu) - p(\lambda) = \left( \mu - \lambda \right) q(\lambda), $$
				für ein $q \in \mathcal{P}$ für alle $\lambda \in \C$.
					$$ \xRightarrow[\lambda = A]{(1), (2)} p(\mu) - p(A) = \left( \mu - A \right) q(A) = q(A) \left( \mu - A \right) $$
				Da $\mu \in \sigma(A)$ ist $\mu - A$ nicht injektiv oder nicht surjektiv.
					$$ \Rightarrow p(\mu) - p(A) $$
				kann nicht injektiv oder nicht surjektiv sein $\Rightarrow p(\mu) \subseteq \sigma \left( p(A) \right)$. \bigskip
				\enquote{$\subseteq$}: Sei $\mu \in \sigma \left( p (A) \right)$.Wähle Nullstellen $\lambda_1, \dotsc, \lambda_m \in \C$ von $\lambda \mapsto \mu - p(\mu)$, d.h.
				$$ \mu - p(\lambda) = a \left( \lambda - \lambda_1 \right) \cdot \dotsc \cdot \left( \lambda - \lambda_m \right), $$
				$a \neq 0 \xRightarrow[\lambda ) A]{(1), (2)} \mu - p(A) = a \left(A - \lambda_1 \right) \cdot \dotsc \cdot \left(A - \lambda_m \right)$. Angenommen: $\lambda_1, \dotsc, \lambda_m \notin \sigma(A)$
				$$ \Rightarrow L(X) \ni \left(A - \lambda_m \right)^{-1} \cdot \dotsc \cdot \left( A - \lambda_1 \right)^{-1} a^{-1} = \left( \mu - p(A) \right)^{-1}, $$
				was einen Widerspruch zu $\mu \in \sigma \left( p(A) \right)$ darstellt $\Rightarrow \exists j_0 \in \{1, \dotsc, m \}$: $\lambda_{j_0} \in \sigma(A)$.
				$$ \mu - p(\lambda_{j_0}) = 0 \iff \mu = p(\lambda_{j_0}), $$
				d.h. $\mu \in p \left( \sigma(A) \right)$.
		\end{enumerate}
	\end{proof}
\end{satz}

\begin{bemerkung} ~\\
	\begin{figure*}[h!] \centering
		\begin{tabular}{ccc}
  			~ & \textbf{Kalkül für Polynome} & ~ \\
  			\hline
  			Verallgemeinerte & & Approximiere: \\
  			Polynome = Potenzreihen & & Satz von Weierstraß \\
  			& ~ & $\overline{\mathcal{P}}^{\|\cdot\|_{C^{0}[0,1]}} = C^{0}[0,1]$ \\
  			$A \mapsto \sum_{k =0}^{\infty} a_k A^k$ & ~ & ~ \\
 			\enquote{Konvergenzradius}? & ~ & $\Rightarrow$ Kalkül für $C^0[0,1]$  \\
  			$\Rightarrow$ analytische Funktionen   & ~ & Stetige Funktionalkalkül \\
  			$\Rightarrow$ Dunford-Kalkül & ~ & ~ 
		\end{tabular}
	\end{figure*}
\end{bemerkung}

\begin{definition}[Kalkül für Potenzreihen]
	Sei $f(\lambda) = \sum_{k=0}^{\infty} a_k \lambda^k$, $\left( a_k \right)_{k \in \N_0} \subseteq \C$, Potenzreihe mit Konvergenzradius $R > 0$. Zu $A \in L(X)$, $r(A) < R$ definieren wir:
	$$ f(A) \coloneqq \lim_{n \rightarrow \infty} \left( \sum_{k = 0}^{n} a_k A^k \right) \eqqcolon \sum_{k =0}^{\infty} a_k A^{k} $$
	in $L(X)$.
\end{definition}


% Exercises - End 

% Index									
\renewcommand{\indexname}{Stichwortverzeichnis}
\printindex


\end{document}