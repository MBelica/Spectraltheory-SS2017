\documentclass[12pt]{extreport} % Schriftgröße: 8pt, 9pt, 10pt, 11pt, 12pt, 14pt, 17pt oder 20pt

%% Packages
\usepackage{scrextend}
\usepackage{amssymb}
\usepackage{amsthm}
\usepackage{amsmath}
\usepackage{dsfont}
\usepackage{booktabs}
\usepackage{chngcntr}
\usepackage{cmap}
\usepackage{color}
\usepackage{csquotes}
\usepackage{enumitem}
\usepackage{float}
\usepackage{bbding}
\usepackage{hyperref}
\usepackage{ulem}
\usepackage{lmodern}
\usepackage{makeidx}
\usepackage{mathtools}
\usepackage{xpatch}
\usepackage{pgfplots}
\pgfplotsset{compat=1.7}
\usetikzlibrary{calc}	
\usetikzlibrary{matrix}	

% Language Setup (English)
\usepackage[utf8]{inputenc} 
\usepackage[T1]{fontenc} 
\usepackage[english]{babel}

% Options
\makeatletter%%  
  % Linkfarbe, {0,0.35,0.35} für Türkis, {0,0,0} für Schwarz, {1,0,0} für Rot, {0,0,0.85} für Blau
  \definecolor{linkcolor}{rgb}{0,0.35,0.35}
  % Zeilenabstand für bessere Leserlichkeit
  \def\mystretch{1.2} 
  % Publisher definieren
  \newcommand\publishers[1]{\newcommand\@publishers{#1}} 
  % Enumerate im 1. Level: \alph für a), b), ...
  \renewcommand{\labelenumi}{\alph{enumi})} 
  % Enumerate im 2. Level: \roman für (i), (ii), ...
  \renewcommand{\labelenumii}{(\roman{enumii})}
  % Zeileneinrückung am Anfang des Absatzes
  \setlength{\parindent}{0pt} 
  % Für das Proof-Environment: 'Beweis:' anstatt 'Beweis.'
  \xpatchcmd{\proof}{\@addpunct{.}}{\@addpunct{:}}{}{} 
  % Nummerierung der Bilder, z.B.: Abbildung 4.1
  \@ifundefined{thechapter}{}{\def\thefigure{\thechapter.\arabic{figure}}} 
  % Chapter-Nummerierung beginnen bei (0):
  \setcounter{chapter}{0}
  % Chapter-Nummerierung
  \renewcommand\thechapter{\Roman{chapter}}
\makeatother%

% Meta Setup 
\title{Spectraltheory}
\author{Prof. Dr. Tobias Lamm}
\date{Sommersemester 2017}
\publishers{Karlsruher Institut für Technologie}

%% Math. Definitiones
\newcommand{\C}{\mathbb{C}}
\newcommand{\N}{\mathbb{N}}
\newcommand{\Q}{\mathbb{Q}}
\newcommand{\R}{\mathbb{R}}
\newcommand{\Z}{\mathbb{Z}}
\newcommand{\DO}[1]{\mathcal{D}\left( {#1} \right)}
\newcommand{\RO}[1]{\mathcal{R}\left( {#1} \right)}

\newtheoremstyle{named}{}{}{\normalfont}{}{\bfseries}{:}{0.25em}{#2 \thmnote{#3}}
\newtheoremstyle{nnamed}{}{}{\normalfont}{}{\bfseries}{:}{0.25em}{\thmnote{#3}}
\newtheoremstyle{itshape}{}{}{\itshape}{}{\bfseries}{:}{ }{}
\newtheoremstyle{normal}{}{}{\normalfont}{}{\bfseries}{:}{ }{}
\renewcommand*{\qed}{\hfill\ensuremath{\square}}

\theoremstyle{named}
\newtheorem{unnamedtheorem}{Theorem} \counterwithin{unnamedtheorem}{chapter}
\theoremstyle{nnamed}
\newtheorem*{unnamedtheorem*}{Theorem} 

\theoremstyle{itshape}
\newtheorem{definition}{Definition}  \counterwithin{definition}{chapter}
\newtheorem{theorem}{Theorem}  \counterwithin{theorem}{chapter}
\newtheorem{lemma}{Lemma}  \counterwithin{lemma}{chapter}

\theoremstyle{normal}
\newtheorem*{definition*}{Definition}
\newtheorem*{theorem*}{Theorem}
\newtheorem*{recall}{Recall}
\newtheorem*{example}{Example}
\newtheorem*{statements}{Statements}
\newtheorem*{remark}{Remark}
\newtheorem*{remarks}{Remarks}
\newtheorem*{satz}{Satz}
\newtheorem*{bemerkung}{Bemerkung}

%% Template
\makeatletter%
\DeclareUnicodeCharacter{00A0}{ } \pgfplotsset{compat=1.7} \hypersetup{colorlinks,breaklinks, urlcolor=linkcolor, linkcolor=linkcolor, pdftitle=\@title, pdfauthor=\@author, pdfsubject=\@title, pdfcreator=\@publishers}\DeclareOption*{\PassOptionsToClass{\CurrentOption}{report}} \ProcessOptions \def\baselinestretch{\mystretch} \setlength{\oddsidemargin}{0.125in} \setlength{\evensidemargin}{0.125in} \setlength{\topmargin}{0.5in} \setlength{\textwidth}{6.25in} \setlength{\textheight}{8in} \addtolength{\topmargin}{-\headheight} \addtolength{\topmargin}{-\headsep} \def\pulldownheader{ \addtolength{\topmargin}{\headheight} \addtolength{\topmargin}{\headsep} \addtolength{\textheight}{-\headheight} \addtolength{\textheight}{-\headsep} } \def\pullupfooter{ \addtolength{\textheight}{-\footskip} } \def\ps@headings{\let\@mkboth\markboth \def\@oddfoot{} \def\@evenfoot{} \def\@oddhead{\hbox {}\sl \rightmark \hfil \rm\thepage} \def\chaptermark##1{\markright {\uppercase{\ifnum \c@secnumdepth >\m@ne \@chapapp\ \thechapter. \ \fi ##1}}} \pulldownheader } \def\ps@myheadings{\let\@mkboth\@gobbletwo \def\@oddfoot{} \def\@evenfoot{} \def\sectionmark##1{} \def\subsectionmark##1{}  \def\@evenhead{\rm \thepage\hfil\sl\leftmark\hbox {}} \def\@oddhead{\hbox{}\sl\rightmark \hfil \rm\thepage} \pulldownheader }	\def\chapter{\cleardoublepage  \thispagestyle{plain} \global\@topnum\z@ \@afterindentfalse \secdef\@chapter\@schapter} \def\@makeschapterhead#1{ {\parindent \z@ \raggedright \normalfont \interlinepenalty\@M \Huge \bfseries  #1\par\nobreak \vskip 40\p@ }} \newcommand{\indexsection}{chapter} \patchcmd{\@makechapterhead}{\vspace*{50\p@}}{}{}{}\def\Xint#1{\mathchoice
    {\XXint\displaystyle\textstyle{#1}} {\XXint\textstyle\scriptstyle{#1}} {\XXint\scriptstyle\scriptscriptstyle{#1}} {\XXint\scriptscriptstyle\scriptscriptstyle{#1}} \!\int} \def\XXint#1#2#3{{\setbox0=\hbox{$#1{#2#3}{\int}$} \vcenter{\hbox{$#2#3$}}\kern-.5\wd0}} \def\dashint{\Xint-} \def\Yint#1{\mathchoice {\YYint\displaystyle\textstyle{#1}} {\YYYint\textstyle\scriptscriptstyle{#1}} {}{} \!\int} \def\YYint#1#2#3{{\setbox0=\hbox{$#1{#2#3}{\int}$} \lower1ex\hbox{$#2#3$}\kern-.46\wd0}} \def\YYYint#1#2#3{{\setbox0=\hbox{$#1{#2#3}{\int}$}  \lower0.35ex\hbox{$#2#3$}\kern-.48\wd0}} \def\lowdashint{\Yint-} \def\Zint#1{\mathchoice {\ZZint\displaystyle\textstyle{#1}}{\ZZZint\textstyle\scriptscriptstyle{#1}} {}{} \!\int} \def\ZZint#1#2#3{{\setbox0=\hbox{$#1{#2#3}{\int}$}\raise1.15ex\hbox{$#2#3$}\kern-.57\wd0}} \def\ZZZint#1#2#3{{\setbox0=\hbox{$#1{#2#3}{\int}$} \raise0.85ex\hbox{$#2#3$}\kern-.53\wd0}} \def\highdashint{\Zint-} \DeclareRobustCommand*{\onlyattoc}[1]{} \newcommand*{\activateonlyattoc}{ \DeclareRobustCommand*{\onlyattoc}[1]{##1} } \AtBeginDocument{\addtocontents{toc} {\protect\activateonlyattoc}} 
	% Titlepage
	\def\maketitle{ \begin{titlepage} 
			~\vspace{3cm} 
		\begin{center} {\Huge \@title} \end{center} 
	 		\vspace*{1cm} 
	 	\begin{center} {\large \@author} \end{center} 
	 	\vspace*{-0.5cm}
	 	\begin{center} \@date \end{center} 
	 		\vspace*{7cm} 
	 	\begin{center} \@publishers \end{center} 
	 		\vfill 
	\end{titlepage} }
\makeatother%

% Create Index
\makeindex 

\begin{document}

\pagenumbering{Alph}
\begin{titlepage}
	\maketitle
	\thispagestyle{empty}
\end{titlepage}

% Table of Contents
\tableofcontents
\thispagestyle{empty}

% Lecture Notes - Start 			
\pagenumbering{arabic}

\chapter{Unbounded, adjoint and self-adjoint Operators}


Let $H$ be a separable Hilbert space, i.e. a real or complex inner product space that is also a complete metric space with respect to the distance function induced by the inner product $\langle \cdot, \cdot \rangle$ on $H$. ~\bigskip

\begin{recall} \index{inner product}
	A mapping $\langle \cdot, \cdot \rangle \colon H \times H \rightarrow \C$
	is called an \textbf{inner product}, if for all $x, y \in H$, $\lambda \in \C$ holds:
	\begin{description}
	 	\item[$\hspace{0.5cm} (S1) \hspace{0.1cm} $] $\langle x_{1} + x_{2}, y \rangle = \langle x_{1}, y \rangle + \langle x_{2}, y \rangle$, $\langle x, y_{1} + y_{2} \rangle = \langle x, y_{1} \rangle + \langle x, y_{2} \rangle$
	 	\item[$\hspace{0.5cm} (S2) \hspace{0.1cm} $] $\langle  \lambda x, y \rangle = \lambda \langle x, y \rangle,$ $\langle x, \lambda y \rangle = \overline{\lambda} \langle x, y \rangle$
	 	\item[$\hspace{0.5cm} (S3) \hspace{0.1cm} $] $\langle  x, y \rangle = \overline{\langle y, x \rangle }$
	 	\item[$\hspace{0.5cm} (S4) \hspace{0.1cm} $] $\langle  x, x \rangle \geq 0, \quad \langle x, x \rangle = 0 \iff x = 0$
	\end{description}	
\end{recall} ~\smallskip

\index{domain} \index{range}
A linear operator $T$ in $H$ is a linear map 
	$$ u \mapsto Tu $$
defined on a subspace $\DO{T}$ of $H$, and we call $\DO{T}$ the \textbf{domain} of $T$. For $T \colon \DO{T} \rightarrow H$ we denote the \textbf{range} of $T$ with
$$ \RO{T} \coloneqq \operatorname{Image}\left( T \right). $$
\newpage
\index{bounded} \index{Operator!bounded}
We say that $T$ is \textbf{bounded} if it is continuous from $\DO{T}$ into $H$, with respect to the topology induced by $H$. We recall that if $\DO{T} = H$ holds, boundedness of a linear operator is equivalent to continuity in $0$, boundedness of $T(U_{(X,\|\cdot\|)})$ in $Y$ and that there $\exists c < \infty$ such that $\| T x \| \leq c \| x \|$, for proof see theorem 3.3 in the \href{https://github.com/MBelica/Funktionalanalysis-WS2015}{functional analysis} course. ~\bigskip

\index{dense}
 From now on, if $\DO{T} \neq H$ we will assume that $\DO{T}$ is \textbf{dense} in $H$, i.e. $\overline{\DO{T}} = H$. If in this case $T$ would be bounded then $T$ has a unique continuous extension to all of $H$, for proof see proposition 5.10 in the \href{https://github.com/MBelica/Funktionalanalysis-WS2015}{functional analysis} course.  As this simplifies many considerations some of the following theorems would be trivial, and hence, we won't focus on bounded operators during this lecture.

\begin{recall} \index{closed} \index{Operator!closed} \index{Graph}
	An operator is called \textbf{closed} if the graph 
	$$ G(T) \coloneqq \big\{ (x, y) \in H \times H ~ \big| ~ x \in \DO{T}, y = Tx \big\} $$
	is closed in $H \times H$.
\end{recall}

\index{closed} \index{Operator!closed}
\begin{definition} \label{i.1:def}
	Let $T \colon \DO{T} \rightarrow H$ be a (linear) operator with $\DO{T}$ dense in $H$. Then $T$ is called \textbf{closed} if for all 
		$$ u_n \in \DO{T}, ~ u_n \rightarrow u \in H \text{ and } T u_n \rightarrow v \in H $$
	follows that
		$$ u \in \DO{T}, ~ v = T u $$
	holds.
\end{definition}

\begin{example} ~\
	\begin{enumerate}
		\item Let $H = L^2(\R^n)$, then $\DO{T_{0}} = C_c^\infty(\R^n)$ is dense in $H$. Define the operator
			$$ T_0 = - \Delta,  $$
			and take $u \in W^{2,2}(\R^n) \setminus C_c^{\infty}(\R^n)$, s.t $u \in L^2(\R^n)$. Due to the density:
			$$ \exists ~(u_n)_{n \in \N} \subseteq C_c^{\infty}(\R^n) \colon ~ u_n \rightarrow u \text{ in } W^{2,2}(\R^n). $$
			As a result, $\left( u_n, -\Delta u_n \right) \in G(T_0)$ converges in $L^2 \times L^2$ to $(u, -\Delta u) \notin G(T_0)$.
		\item Let $H = L^2(\R^n)$, and set $\DO{T_1} = W^{2,2}(\R^n) \subseteq H$. Define the operator
			$$ T_1 = - \Delta. $$
		 For $(u_n)_{n \in \N} \subseteq \DO{T_1}$ with 
			$$ u_n \rightarrow u \in H \text{ and } \left( - \Delta u_n \right) \rightarrow v \in L^2 $$
			follows that $- \Delta u = v \in L^2(\R^n)$ weakly, i.e. for all $\varphi \in C_c^{\infty}(\R^n)$:
			$$ \int_{\R^n} v \varphi \longleftarrow \int_{\R^n} \left( - \Delta u_n \right) \varphi = \int_{\R^n} u_n \left( - \Delta \varphi \right) \longrightarrow \int_{\R^n} u \left( - \Delta \varphi \right). $$
			$\xRightarrow[]{PDE} u \in W^{2,2}(\R^n) = \DO{T_1} \Rightarrow T_1$ is closed.
	\end{enumerate}
\end{example}

\index{closable} \index{Operator!closable}
\begin{definition}
	An operator $T$ is called \textbf{closable} $\iff \overline{G(T)}$ is a graph.
\end{definition}

\index{closure} \index{Operator!closure of}
\begin{remark}
	We call $\overline{T}$ the \textbf{closure} of $T$, and in such case we have 
	$$ \DO{\overline{T}} \coloneqq \big\{ x \in H ~ \big| ~ \exists ~ y \colon (x, y) \in \overline{G(R)} \big\} $$	
	For any $x \in \DO{\overline{T}}$ the assumption that $\overline{G(T)}$ is a graph implies that $y$ is unique and hence
	$$ \Rightarrow G(\overline{T}) = \overline{G(T)}, ~\overline{T} x \coloneqq y $$
	Equivalently, $\DO{\overline{T}}$ is the set of all $x \in H$ such that there exists a sequence $x_n \in \DO{T}$ with $x_n \rightarrow x$ in $H$ and $T x_n$ is a cauchy sequence. For such $x$ we define
	$$ \overline{T}x \coloneqq \lim_{n \rightarrow \infty} T x_n $$
\end{remark}

\begin{example}
	Let $T_0 \coloneqq - \Delta$, $\DO{T_0} = C_c^\infty(\R^n)$ is closable with $\overline{T_0} = T_1$.
	 
	\begin{proof}
		Let $u \in L^2(\R^n)$ such that there exists $(u_n)_{n \in \N} \subseteq C_c^\infty(\R^n)$ with $u_n \rightarrow u$ in $L^2$ and $-\Delta u_n \rightarrow u$ in $L^2$, as above: 
		$$ -\Delta u = v \in L^2 $$
		For a given $u$ the function $v$ is unique, and hence, $T_0$ is closable. Let $\overline{T_0}$ be the closure with domain $\DO{\overline{T_0}}$ and $u \in \DO{T_0}$
		$$ \Rightarrow \Delta u \in L^2 \Rightarrow u \in W^{2,2}(\R^n) = \DO{T_1} \Rightarrow \DO{\overline{T_0}} \subseteq \DO{T_1} $$
		but $C_c^{\infty}(\R^n)$ is dense in $\DO{T_1} = W_{2,2}(\R^n)$
		$$ \Rightarrow W^{2,2}(\R^n) \subseteq \DO{\overline{T_0}} \Rightarrow \DO{\overline{T_0}} = \DO{T_1} $$
		$\Rightarrow T_1 = \overline{T_0}$. 
	\end{proof}	
\end{example}

\begin{remark}
	Assume for a second in the example above that $W^{2,2} \not\subseteq \DO{T_0}$ holds:
		$$ \Rightarrow \exists u \in W^{2, 2} \setminus D \left( \overline{T_0} \right), ~\exists (u_n)_{n \in \N} \in C^{\infty}_c (\R^n): u_n \rightarrow u \text{ in } W^{2, 2}, $$
		using the same arguments as in the example above $\Rightarrow \overline{T_0}$ not closed!	
\end{remark}


\begin{recall}
	If $T \colon H \rightarrow H$ is bounded then $T^*$ is defined through
	$$ \langle u , T^* v \rangle = \langle T u , v \rangle, ~ \forall u, v \in H $$	
	$u \mapsto \langle T u, v \rangle$ defines a continuous linear map on $H$ ($\in H'$). Riesz' representation theorem then ensures the existence of $T^*$.
\end{recall}

\index{adjoint}
\begin{definition}
	If $T$ is an unbounded operator on $H$ with dense domain we define
		$$ \DO{T^*} \coloneqq \big\{ v \in H \colon \DO{T} \ni u \mapsto \langle T u, v \rangle \text{ can be extended as a linear continuous form on } H \big\} $$
	Using Riesz' representation theorem $\exists! f \in H$:
		$$ \langle u, f \rangle = \langle T u, v \rangle, ~ \forall u \in \DO{T} $$
	then define $T^* v = f$, where the uniqueness follows from the density of $\DO{T}$ in $H$.
\end{definition}

\begin{remark}
	If $\DO{T} = H$ and $T$ is bounded then we recover the \enquote{old} adjoint.
\end{remark}

\begin{example}
	$T_0^* = T_1$, 
	\begin{align*}
		\DO{T_0^*} & = \big\{ v \in L^2(\R^n) \colon C_c^\infty(\R^n) \ni u \mapsto \langle - \Delta u, v \rangle \text{ extendable as a lin. continuous form on } L^2(\R^n) \big\} \\
			& = \big\{ v \in L^2(\R^n) \colon - \Delta v \in L^2(\R^n) \big\} = W^{2,2}(\R^n) = \DO{T_1}
	\end{align*}
	Damit ist
	$$ \langle T_1 u, v \rangle = \langle - \Delta u, v \rangle = \int v \left( - \Delta u \right) = \int \left( - \Delta v \right) u = \langle u, T_1 v \rangle $$
\end{example}

\begin{theorem} \label{I.1:thm}
	$T^*$ is a closed operator.
	
	\begin{proof}
		$v_n \in \DO{T^*}$ such that $v_n \rightarrow v$ in $H$ and $T^* v_n \rightarrow w^*$ in $H$ for $(v, w^*) \in H \times H$. ~\\
		For all $u \in \DO{T}$ we have
			$$ \langle Tu, v \rangle = \lim_{n \rightarrow \infty} \langle T u, v_n \rangle = \lim_{n \rightarrow \infty} \langle u, T^* v_n \rangle = \langle u , w^* \rangle $$
		($H \ni u \mapsto \langle u, w^* \rangle$ is continuous) $\Rightarrow v \in \DO{T^*}$ and $w^* = T^* v$ by definition.
	\end{proof}
\end{theorem}

\begin{theorem} \label{I.2:thm}
	Let $T$ be an operator in $H$ with domain $\DO{T}$. Then
		$$ G \left( T^* \right) = \left( V \left( \overline{G(T)} \right) \right)^{\perp} $$
	where $V \colon H \times H \rightarrow H \times H, V(x, y) = (y, -x)$ ($V^2 = - \mathds{1}$).
	
	\begin{proof}
		Let $u \in \DO{T}, \left(v, w^* \right) \in H \times H$ 
			$$ \Rightarrow \langle V \left( u, Tu \right), \left( v, w^* \right) \rangle_{H \times H} = \langle T u, v \rangle - \langle u, w^* \rangle $$
		Considering the right-hand side it follows
		$$ \langle T u, v \rangle - \langle u, w^* \rangle = 0 ~ \forall u \in \DO{T} \iff v \in \DO{T^*} \text{ and } w^* = T^* v \iff \left( v, w^* \right) \in G\left( T^* \right), $$
		and considering the left-hand side:
		$$ \Rightarrow \langle V \left( u, Tu \right), \left( v, w^* \right) \rangle_{H \times H} = 0 ~ \forall u \in \DO{T} \iff \left( v, w^* \right) \in V \left( G(T) \right)^{\perp} $$
		In general: $U^{\perp} = \overline{U}^{\perp}$, and hence
		$$ \Rightarrow V \left( G(T) \right)^{\perp} = \left( \overline{V\left( G(T) \right)} \right)^{\perp} = \left( V \left( \overline{G(T)} \right) \right)^{\perp}. $$
	\end{proof}
\end{theorem}

\begin{theorem} \label{I.3:thm}
	Let $T$ be a closable operator. Then:
	\begin{enumerate}
		\item $\DO{T^*}$ is dense in $H$
		\item $T^{**} \coloneqq \left( T^* \right)^* = \overline{T}$
	\end{enumerate} 
	
	\begin{proof} ~\
		\begin{enumerate}
			\item Proof through contradiction: $D\left( T^* \right)$ not dense in $H \rightarrow \exists w \neq 0: \langle w, v \rangle = 0 ~\forall v \in \overline{\DO{T^*}}$
				\begin{align*}
					& \xRightarrow[]{~~~~} \langle \left( 0, w \right), \left( T^*v, -v \right) \rangle_{H \times H} = 0 ~\forall v \in \DO{T^*} \\
					& \xRightarrow[]{~~~~} \left( 0, w \right) \perp V \left( G(T^*) \right) \\
					& \xRightarrow[\ref{I.2:thm}]{\hyperref[I.2:thm]{Thm}} V \left( \overline{G(T)} \right) = G \left( T^* \right)^{\perp} \\
					& \xRightarrow[]{~~~~} V \left( G(T^*)^{\perp} \right) = \overline{G(T)}
				\end{align*}
				For any $M \subseteq H \times H$ we have $V \left( M^{\perp} \right) = V(M)^{\perp}$ since for $(u,v) \in V(M)^{\perp}$, $(x,y) \in M$
				$$ \langle V(u, v) , (x,y) \rangle_{H \times H} = - \langle (u, v ), V(x,y) \rangle_{H \times h} \Rightarrow V(u, v) \in M^{\perp} \Rightarrow (u, v) \in V\left( M^{\perp} \right) $$
				$\Longrightarrow V \left( G \left( T^* \right) \right)^{\perp} = \overline{G(T)} = G \left(\overline{T} \right) \Longrightarrow (0, w) \in G\left( \overline{T} \right) \Longrightarrow w = 0$
			\item $G \left( T^{**} \right) \overset{\hyperref[I.2:thm]{Thm}}{\underset{\ref{I.2:thm}}{=}} V \left( \overline{G \left( T^* \right)} \right)^{\perp} \overset{\hyperref[I.1:thm]{Thm}}{\underset{\ref{I.1:thm}}{=}} V \left( G \left( T^* \right) \right)^{\perp} \overset{(\perp)}{=} G \left( \overline{T} \right) \Longrightarrow \DO{T^{**}} = \DO{\overline{T}}, T^{**} = \overline{T}$
		\end{enumerate}
	\end{proof}
\end{theorem}

\index{symmetric}
\begin{definition}
	We say $T \colon \DO{T} \rightarrow H$ is \textbf{symmetric} if and only if
	$$ \langle Tu, v \rangle = \langle u , Tv \rangle \quad \forall u, v \in \DO{T} $$
\end{definition}

\begin{example}
	$T_0 = - \Delta$, $\DO{T_0} = C_c^{\infty}(\R^n)$
	$$ \int_{\R^n} \left( - \Delta u \right) v = \int_{\R^n} u \left( - \Delta v \right) $$	
\end{example}

\begin{remark}
	If $T$ is symmetric $\Rightarrow \DO{T} \subseteq \DO{T^*}$ and
	$$ T u = T^* u \quad \forall u \in \DO{T} $$
	$\Rightarrow \left( T^*, \DO{T^*} \right)$ is an extension of $\left( T, \DO{T} \right)$.
\end{remark}

\begin{lemma} \label{I.1:lmm}
	A symmetric operator $T$ is closable.
	
	\begin{proof}
		It suffice to show that for $u_n \in \DO{T}$ with $u_n \rightarrow 0$ and $ T u_n \rightarrow x \in H$ we have $x = 0$
		$$ \langle x,  v \rangle \leftarrow \langle T u_n, v \rangle = \langle u_n , T v \rangle \rightarrow \langle 0, T u \rangle = 0 \quad \forall v \in \DO{T}  $$
		$\Rightarrow x = 0$.
	\end{proof}
\end{lemma}

\begin{remark}
	The proof actually shows that if $\DO{T^*}$ is dense in $H$, then $T$ is closable.	
\end{remark}

\index{self-adjoint}
\begin{definition}
	We call an operator $T$ \textbf{self-adjoint} if 
		$$ T = T^* \text{ and }  Tu = T^* u \quad \forall u \in \DO{T},  $$ % todo wieso ist diese Forderung nötig?
	note that the first property implies that $\DO{T} = \DO{T^*}$.
\end{definition}

\begin{theorem} \label{thm:1.4}
	Every self-adjoint operator is closable.
	
	\begin{proof}
		\hyperref[I.1:lmm]{Lemma I.1}
	\end{proof}
\end{theorem}

\begin{theorem}
	Let $T$ be an invertible self-adjoint operator, then $T^{-1}$ is also self-adjoint.
	
	\begin{proof}
		For $T \colon \DO{T} \rightarrow \RO{T}$ consider
		\begin{description}
			\item[~\hspace{0.25em}~Step 1] $\RO{T}$ is dense in $H$. We have to show that $\RO{T}^\perp = \{ 0 \}$. ~\\
				Let $w \in H$ such that 
					$$\langle T u, w \rangle = 0 ~\forall u \in \DO{T}$$
				$\Longrightarrow w \in \DO{T^*}$ and $T^* w = 0 \xRightarrow[s.a.]{inj.} w = 0$.
			\item[~\hspace{0.25em}~Step 2] Let $w \colon H \times H \rightarrow H \times H$, $w(x,y) = (y, x)$
			$$ \Longrightarrow G\left( T^{-1} \right) = \big\{ \left(x, T^{-1} x\right) \colon x \in \DO{T} \big\} = w\left(G\left(T\right)\right) = \big\{ \left(Ty, y\right) \colon y \in \DO{T} \big\} $$
			\begin{align*}
				G\left( T^{-1} \right) & = G \left( \left( T^* \right)^{-1} \right) ~\overset{\hyperref[I.2:thm]{Proof}}{\underset{\hyperref[I.2:thm]{Thm. I.2}}{=}} w \left( V \left( G \left(T \right)^{\perp} \right) \right) \\
				& = V \left( w \left( G \left( T \right) \right)^{\perp} \right) =  V \left( w \left( G \left( T \right) \right) \right)^{\perp} \\
				& =  V \left(  G \left( T^{-1} \right) \right)^{\perp} \overset{\hyperref[I.2:thm]{Thm.}}{\underset{\ref{I.2:thm}}{=}} G \left( \left( T^{-1} \right)^* \right)
			\end{align*}
			$\Rightarrow T^{-1} = \left( T^{-1} \right)^*$
		\end{description}
	\end{proof}
\end{theorem}


\chapter{Representation Theorems}

\begin{theorem}[Riesz]
	Let $u \mapsto F(u)$ be a linear continuous function on $H$. Then $\exists! w \in H$:
	$$ F(u) = \langle u, w \rangle \quad \forall u \in H $$
\end{theorem}

\textbf{Lax-Milgram}: $V$ Hilbertspace, sesquilinear form is defined on $V \times V$, $(u, v) \mapsto \alpha (u, v)$ continuous with
$$ \left| \alpha (u, v) \right| \leq c \| u \| \|v \| \quad \forall u,v \in V $$
\textbf{Riesz}: $\exists$ linear map $A \colon V \rightarrow V$:
$$ \alpha(u, v ) = \langle A u, v \rangle $$

\index{coercive}
\begin{definition}
	A bilinear form $a \colon V \times V \rightarrow \R$ is \textbf{$V$-coercive} if there exists $\lambda > 0$ such that
	$$ a\left( u, u \right) \geq \lambda \| u \|^2 \quad \forall u \in V $$
\end{definition}

\begin{theorem}
	Let $a$ be a continuous sesquilinear and $V$-coercive on $V \times V$ then $A$ is an isomorphism.
	
	\begin{proof} ~\
		\begin{description}
			\item[~\hspace{0.25em}~Step 1:] $A$ is injective:
				\begin{align*}
					\| Au \| \| u \| \overset{C.S.}{\geq} \left| \langle A u, u \rangle \right| = \left| a(u, u) \right| \geq \lambda \| u \|^2 \tag*{$(+)$}
				\end{align*}  
				$\Rightarrow \| A u\| \geq \lambda \| u \|$ for all $u \in V$.
			\item[~\hspace{0.25em}~Step 2:] $A(V)$ is dense in $V$. Let $u \in V$ such that 
				$$ \langle A u, v \rangle = 0 \quad \forall v \in V $$
				take $v = u \Rightarrow a(u, u) = 0 \Rightarrow u = 0$. 
			\item[~\hspace{0.25em}~Step 3:] $\R(A) = A(V)$ is closed. Let $v_n$ be a sequence in $A(V)$ and let $u_n$ be such that
				$$ A u_n = v_n $$
				$\xRightarrow[]{(+)} u_n$ is a Cauchy sequence $\Rightarrow u_n \rightarrow u \in V$ und $A u_n \rightarrow A u \Rightarrow v_n \rightarrow A u \in A(V)$
			\item[~\hspace{0.25em}~Step 4:] $u = A^{-1} v \xRightarrow[]{(+)} \|A^{-1} v \| \leq \lambda^{-1} \| v \| ~\forall v \in V$. 
		\end{description}
	\end{proof}
\end{theorem}

Next we consider two Hilbert spaces $V, H$ with $V \subset H$ (the inclusion is continuous), i.e.
$$ \exists c < \infty \colon \quad \| u \|_H \leq c \| u \|_V \quad \forall u \in V $$
and we assume that $V$ is dense in $H$.

\begin{example}
	$V = W^{1,2}(\R^n)$, $H = L^2(\R^n)$
 		$$ \| u \|_L^2 \leq \| u \|_{W^{1,2}} $$	
 	There exists a natural injection from $H$ into $V'$. Let $h \in H$ then $V \ni u \mapsto \langle u, h \rangle_H$ is continuous on $V$ $\xRightarrow[\text{\hyperref[II.1:thm]{Thm. II.1}}]{} \exists l_h \in V'$:
 		$$ l_h(u) = \langle u, h \rangle_H \quad \forall u \in V $$
 	injectivity follows from density of $V$ in $H$. $V \subseteq H \subset V'$ cont. sesquilinear form $a$ on $V \times V$ which is $V$-coercive $\rightarrow$ Associate an unbounded operator $S$ with $a$
 		$$ \DO{S} \coloneqq \big\{ u \in V \colon a(u, v) \text{ is cont. on } V \text{ with respect to the topology induced by } H \big\} $$
\end{example}

\begin{center}
	\textit{\color{red} skipped a part! Unreadable}
\end{center} % todo skipped a part! Unreadable


\begin{theorem} \label{thm:2.3}
	Let $a$ be a continuous sesquilinear form on $V$ which is $V$-coercive then $S$ is bijective from $\DO{S}$ into $H$ and $S^{-1} \in L(H, \DO{S})$. Moreover, $\DO{S}$ is dense in $H$.
	
	\begin{proof} ~\
		\begin{enumerate}[label=\arabic*\upshape)]
			\item $S$ injective: $\exists \alpha > 0$: 
				$$ \alpha \| u \|^2_H \leq C \alpha \| u \|^2_V \leq C \left| a (u,u) \right| = c \left| \langle S u, u \rangle_H \right| \leq c \| Su\|_H \| u \|_H, \quad \forall u \in \DO{S} $$
				$\Rightarrow \alpha \|u\|_H \leq c \| Su\|_H, ~ \forall u \in \DO{S} ~(+)$.
			\item $S$ surjective: ~\\
				Let $h \in H$. Choose $w \in V$ such that
				$$ \langle h, v \rangle_H = \overline{\langle v, h \rangle_H} = \overline{l_h(v)} = \langle w, v \rangle \quad \forall v \in V $$
				where we used Riesz' representation theorem in the last step. 
				$$\text{(Note: $l_h \in V' \Rightarrow \overline{l_h} \in$ continuous lineare form on $V$).} $$  
				\begin{center}
					\textit{\color{red} is tis correct? Hard to read!}
				\end{center} % todo stimmt das hier? Schwer lesbar!
				Define $u \coloneqq A^{-1} w \in V \Rightarrow a(u,v) = \langle A u, v \rangle_V = \langle w, v \rangle_V = \langle h, v \rangle_H$
				$$ \Rightarrow u \in \DO{S}, ~Su = h $$
				($V$ dense in $H$). $(+)$ implies that $S^{-1}$ is continuous.
			\item Density of $\DO{S}$: ~\\
				Let $h \in H$ such that $\langle u, h \rangle_H = 0 ~\forall u \in \DO{S}$. $S$ surjective $\exists v \in DO{S}$: $Sv = h$
				$$ \Rightarrow \langle Sv, u \rangle = 0 ~ \forall u \in \DO{S} $$
				$$ \Rightarrow \langle Sv, v \rangle_H = 0 \Rightarrow a(v, v) = 0 \Rightarrow v = 0 \Rightarrow h = 0 $$
				$a$ hermitian iff
				$$ a(u,v) = \overline{a(v, u)} \quad \forall u, v \in V $$
		\end{enumerate}
	\end{proof}	
\end{theorem}

\begin{theorem}
	Under the assumptions of \hyperref[thm:2.3]{Theorem II.3} and $a$ being hermitian it follows that
		\begin{enumerate}
			\item $S$ is closed
			\item $S = S^*$
			\item $\DO{S}$ dense in $V$
		\end{enumerate}
		
		\begin{proof} ~\
			\begin{enumerate}
				\item \hyperref[thm:1.4]{Theorem I.4}
				\item $a$ hermitian
					$$ \Rightarrow \langle Su, v \rangle_H = a(u, v) = \overline{a(v, u)} = \overline{\langle Sv, u \rangle_H} = \langle u, Sv \rangle_H \quad \forall u, v \in \DO{S} $$
					$\Rightarrow S$ symmetric $\Rightarrow \DO{S} \subset \DO{S^*}$. Let $v \in \DO{S^*}$, $S$ surjective 
						$$\Rightarrow v_0 \in \DO{S}: S v_0 = S^* v. $$ 
					For all $u \in \DO{S}$ we get
					$$ \langle Su, v_0 \rangle_H = \langle u, S v_0 \rangle_H = \langle u, S^* v \rangle_H = \langle S u, v \rangle_H $$
					$\Rightarrow v = v_0 \Rightarrow \DO{S} = \DO{S^*}$, $Sv = S^* v ~\forall v \in \DO{S}$.
				\item follows from \hyperref[thm:2.3]{Theorem II.3}
			\end{enumerate}	
		\end{proof}
\end{theorem}

\chapter{Friedrichs Extension}

\index{semi-bounded} \index{bounded!semi-}
\begin{definition}
	Let $T_0$ be a symmetric unbounded operator with domain $\DO{T_0}$ we say that $T_0$ is \textbf{semi-bounded} if $\exists c > 0$:
		$$ \langle T_0 u, u \rangle_H \geq - c \| u \|_H^2 \quad \forall u \in \DO{T_0} $$	
\end{definition}

\index{Hardy inequality}
\begin{example} ~\
	\begin{enumerate}
		\item Schrödinger Operator. $\R^m$, $H = L^2(\R^m)$, $\DO{T_0} = C_c^\infty(\R^m)$
				$$ T_0 \coloneqq - \Delta + V(x), $$
			$V \in C_0(\R^m)$ with $V(x) \geq - c ~\forall x \in \R^m$. For $u \in \DO{T_0}$
			$$ \langle T_0 u, u \rangle_H = \int_{\R^m} \left( \Delta u + V u \right) u = \underbrace{\int_{\R^m} \left| \nabla u \right|^2}_{\geq 0} + \underbrace{\int_{\R^m} V(x) \left| u(x) \right|^2}_{\geq - c \int |u|^2 = - c \|u \|^2_H} $$
		\item $S_z \coloneqq - \Delta - \frac{z}{r}$, whereas $r = |x|, z \in \R$ ~\\ 
		
			\textbf{Hardy inequality} in $\R^3 (m = 3)$:
				$$ \int_{\R^3} |x|^{-2} |u(x)|^2 dx \leq  4 \int_{\R^3} \left| \nabla u \right|^2 (x) dx \quad \forall u \in C_c^\infty(\R^m) $$
				\begin{proof}
					$\int_{\R^3} \left| \nabla u + \frac{1}{2} \frac{x}{|x|^2} u \right|^2 dx \geq 0$ 
					$$ \iff \int_{\R^3} \left| \nabla u \right|^2 + \frac{1}{4} \frac{|u|^2}{|x|^2} dx \geq - \int_{\R^3} \langle \nabla u(x), \frac{x}{|x|} \rangle u(x) dx $$
					now
					$$ -2 \int_{\R^3} \langle \nabla u, \frac {|x|^2}{|x|^2} \rangle u dx = - \int_{\R^3} \langle \nabla |u|^2, \frac{x}{|x|^2} dx = \int_{\R^3} |u|^2 \underbrace{\operatorname{div} \frac{x}{|x|^2}}_{= \frac{1}{|x|^2}} = \int_{\R^3} \frac{|u|^2}{|x|^2} dx $$ 
				\begin{center}
					\textit{\color{red} is tis correct? Hard to read!}
				\end{center} % todo hier stimmt doch was nicht
					$\Rightarrow \int |\nabla u|^2 \geq \frac{1}{4} \int \frac{|u|^2}{|x|^2}$. Now
					$$ \int_{\R^3} \frac{1}{r} |u(x)|^2 dx \leq \left( \int \frac{|u(x)|^2}{r^2} dx \right)^{\frac{1}{2}} \cdot \| u \|_L^2 $$
					$\int_{\R^3} \frac{1}{r^2} |u(x)|^2 dx \leq 4 \langle - \Delta u, u \rangle_L^2$
					$$ \Rightarrow \forall \epsilon > 0: \quad \int_{\R^3} \frac{1}{r} |u(x)|^2 dx \leq \epsilon \cdot \langle-\Delta u, u \rangle_L^2 + \frac{1}{\epsilon} \| u \|_{L^2}^2 $$
					hence 
					$$ \langle S_z u, u \rangle_{L^2} = \langle - \Delta u, u \rangle_{L^2} - z \langle \frac{u}{r}, u \rangle_{L^2} \geq (1 - \epsilon) \langle - \Delta u, u \rangle_{L^2} - \frac{z}{\epsilon} \| u \|^2_{L^2} $$
					Choose $\epsilon = \frac{1}{z} \Rightarrow \langle S_z u, u \rangle_{L^2} \geq - z^2 \| u\|^2_{L^2}$
				\end{proof}
	\end{enumerate}	
\end{example}

\index{Friedrichs extension}
\begin{theorem}
	A symmetric semibounded operator $T_0$ on $H$ with dense domain $\DO{T_0}$ admits a self-adjoint extension, called \textbf{Friedrichs extension}.
	
	\begin{proof}
		Replace $T_0$ by $T_0 + \lambda \mathds{1}$ such that
		$$ \langle T_0 u, u \rangle_H \geq \|u\|_H^2 \quad \forall u \in \DO{T_0} $$
		$(u,v) \mapsto a_0(u,v) \coloneqq \langle T_0 u, v \rangle_H$ on $\DO{T_0} \times \DO{T_0}$
		$$ \Rightarrow a_0(u,u) \geq \| u \|^2_H $$
		Let $V$ be the completion in $H$ of $\DO{T_0}$ for the norm $u \mapsto \rho_0(u) = \sqrt{a_0(u,u)}$
		$\iff u \in H$ belongs to $V$ if $\exists u_n \in \DO{T_0}$ s.t. $u_n \rightarrow u$ in $H $ and $ (u_n)_{n \in \N}$ is a Cauchy sequence with respect to $\rho_0$ ~\\
		
		\enquote{Candidate Norm}:
		$$ \| u \|_V = \lim_{n \rightarrow \infty} \rho_0(u_n) $$
		where $u_n$ is as above.
	\end{proof}
\end{theorem}

\begin{lemma}
	Let $(x_n)_{n \in \N}$ be a Cauchy sequence in $\DO{T_0}$  with respect to $\rho_0$ such that $x_n \rightarrow 0 $ in $H$. Then $p_0(x_n) \rightarrow 0$. 
	
	\begin{proof}
		Observe that $p_0(x_n)$ is a Cauchy sequence in $\R_+$, and hence, converges in $\overline{\R_+}$. Assume that  $p_0(x_n) \rightarrow \alpha > 0$. Now $a_0(x_n, x_m) = a_0(x_n, x_n) + a_0(x_n, x_m - x_n)$
		$$ \left| a_0(x_n, x_m - x_n) \right| \leq \sqrt{a_0(x_n, x_n)} \sqrt{a_0(x_m - x_n, x_m - x_n)} $$
		$\forall \epsilon > 0 ~\exists N ~\forall n, m \geq N$:
		$$ \left| a_0(x_n, x_m) - \alpha^2 \right| < \epsilon $$
		$\epsilon = \frac{\alpha^2}{2} \Rightarrow |a_0(x_n, x_m) | = \left| \langle T_0 x_n, x_m \rangle \right| \geq \frac{1}{2} \alpha^2 > 0 ~\forall n, m \geq N$. Let $m \rightarrow \infty \Rightarrow x_m \rightarrow 0$, which leads to the contradiction.
	\end{proof}
\end{lemma} % 

				\begin{center}
					\textit{\color{red} page 13 hard to read AND understand structure! skipped that completely}
				\end{center} % todo page 13!??!? skipped that completely
\begin{theorem}[Example: Dirichlet Realisation]
	Let $\Omega \subseteq \R^n$ be a bounded domain, $\partial \Omega$ smooth, $T_1$ is defined by:
	$$ \DO{T_1} = W^{2,2}(\Omega) \cap W_{0}^{1,2}(\Omega), ~ T_1 = - \Delta \colon \DO{T_1} \rightarrow L^2(\Omega) $$
	$T_1$ is self-adjoint; $T_1$ is called the Dirichlet realisation of $-\Delta$.
	
	\begin{proof}(with gaps) ~\\
		Define 
		$$ \DO{T_0} \coloneqq C_c^\infty(\Omega), ~ T_0 \coloneqq - \Delta, ~ H = L^2(\Omega) $$
		Through integration by parts respectively Green's formula:
		$$ T_0 \text{ is symmetric, non-negative (i.e. semi-bounded)} $$
		Consider $\tilde{T_0} \coloneqq T_0 + \mathds{1}_H$. $V$: closure of $C_c^\infty(\Omega)$. in $W^{2,2}(\Omega)$
		$$ \xRightarrow[\underset{ext.}{Friedrich}]{S} \DO{S} = \big\{ u \in W_0^{12}(\Omega) ~| -\Delta u \in L^2(\Omega) \big\} $$
		$$ \xRightarrow[Theory]{Regularity} \DO{S} = W^{2,2} \cap W_0^{1,2}(\Omega) $$
		for more details about the regularity theory see \enquote{2nd order elliptic operators}, PDE Evans.
	\end{proof}
\end{theorem} \newpage

\begin{example} ~\
	\begin{enumerate} 
		\item Harmonic oscillator ~\\
			Define $\DO{T_0} \coloneq C_c^\infty(\R^n)$, $T_0 \coloneqq -\Delta + |x|^2 + 1$, $ = L^2(\R^2)$. Let $V$ be the completion of $C_c^\infty$ in $L^2$ with respect to the norm
			$$ u \mapsto \left( \langle \nabla u, \nabla u \rangle_{L^2} + \langle |x|u(x), |x|u(x)\rangle_{L^2} + \langle u, u \rangle_{L^2} \right)^{\frac{1}{2}} = \sqrt{\langle T_0 u, u \rangle_{L^2}}. $$
			By calculation:
			$$ V = \big\{ u \in W^{1,2}(\R^n) ~|~x_j u \in L{^2}(\R^n) ~\forall j = 1, \dotsc, n \big\} $$
			Domain of $S$:
			$$ \DO{S} = \big\{ u \in V ~|~ T \circ u \in L^2(\R^n) \big\} = \big\{ u \in W^{2,2}(\R^n) \colon x^{\alpha} u \in L^2(\R^n) ~\forall \alpha \in \N_0^n, ~|\alpha| \leq 2 \big\} $$
		\item Schrödinger operator with a Coulomb potential ~\\
			Define $\DO{T_{0}} \coloneqq C_c^\infty(\R^3), ~ T_0 = - \Delta - \frac{1}{r}, ~ H = L^2(\R^3)$. We saw that $T_0$ is semi-bounded: $\langle T_0 u, u \rangle_{L^2} \geq - \| u \|_{L^2}^2$
			$$\tilde{T_0} \coloneqq T_0 + 2 \cdot \mathds{1}_H, ~ \DO{\tilde{T_0}} = \DO{T_0}$$ 
			satisfies the assumptions of the Friedrichs extension. Completion $V$ of $C_c^{\infty}$ in $L^2$ with respect to the norm
			$$ u \mapsto \left( \langle \nabla u, \nabla u \rangle_{L^2} + \int_{\R^3} \left( 2 - \frac{1}{r} \right) |u(x)|^2 dx \right)^\frac{1}{2} = \sqrt{ \langle \tilde{T_0} u, u \rangle_{L^2}} $$
			is $V = W^{1,2}(\R^3)$.
			
			\begin{proof}
				$C_c^\infty(\R^3)$ is dense in $W^{1,2}(\R^3)$. Therefore, we only need to check that the norm above and $\| \cdot \|_{W^{1,2}}$ are equivalent. By the proof of the analysis of the Schrödinger operator:
				$$ \int_{\R^3} \frac{1}{r} |u(x)|^2 dx \leq \epsilon \langle - \Delta u, u \rangle_{L^2} + \frac{1}{\epsilon} \| u \|_{L^2}^2 = \epsilon \| \nabla u \|_{L^2}^2 + \frac{1}{\epsilon} \| u \|^2_{L^2} \quad \forall \epsilon > 0, ~u \in C_c^\infty(\R^3) $$
				(See Bernhard Helfer, \enquote{Spectral theory and app.}).
				$$ \| w \|_{W^{1,2}}^2 = \| \nabla u \|_{L^2}^2 + \| u \|_{L^2}^2 \overset{Hardy}{\leq} 5 \| \nabla u \|^2_{L^2} + \| \left( 2 - \frac{1}{r} \right) u \|^2_{L^2} \quad \forall u \in C_c^\infty(\R^3) $$
				Domination of $S$: Hardy inequality $\Rightarrow \frac{1}{r} u \in L^2$ for $u \in W^{1,2}(\R^3)$.
				$$ \Rightarrow u \in D(S): ~ \Delta u \in L^2(\R^3) \Rightarrow \DO{S} = W^{2,2}(\R^3) $$
			\end{proof}
		\item Neumann boundary conditions: on the half-plane $H = L^2\left( (0, \infty) \right)$ define the form
			$$ a(u, v) \coloneqq \int_{0}^{\infty} u'(x) v'(x) dx  $$
			for $u,v \in \DO{a} = W^{1,2}(0, \infty) \Rightarrow a(u,u) = \|u'\|_{L^2}^2 \geq - \| u \|_{L^2}^2$. $a$ is closed by completeness of $W^{1,2}(0, \infty)$. ~\\
			
			Associated operator $T$: $v \in \DO{T} ~\exists f_v \in L^2(0, \infty)$:
			$$ \int_0^\infty u'(x) v'(x) dx = \int_0^\infty u(x) f_v(x) dx \quad \forall u \in W^{1,2}(0, \infty). $$
			$\Rightarrow f_v = - \left( v' \right)' = - v''$, therefore $v \in W^{2,2}(0, \infty), Tv = - v''$. Note for $v \in W^{2,2}(0, \infty)$, $u \in W^{1,2}(0, \infty)$:
			\begin{align*}
				a(u, v) & = \int_0^\infty u'(x) v'(x) dx \\
				& = \left[ u(x) v'(x) \right]_0^\infty - \int_0^\infty u(x) v''(x) dx \\
				& = \underbrace{u(0)v'(0)}_{= 0} + \int_0^\infty u(x) T v(x) dx = \langle u, Tv \rangle_{L^2}
			\end{align*}
			Thereforem the associated operator is $T_N \coloneqq T$ acts as $T_N v = - v''$ on the domain 
				$$ \DO{T_N} = \big\{ v \in W^{2,2}(0, \infty) ~|~v'(0) = 0 \big\} $$
				$T_N$ is called the Neumann Laplacian.
	\end{enumerate}	
\end{example}

\chapter{Spectrum and Resolvent}

Let $X$ be a Banach space and $H$ a Hilbert space.

\index{resolvent set} \index{spectrum} \index{spectrum!point} \index{spectrum!continuous} \index{spectrum!residual} \index{resolvent function}
\begin{definition}
	Let $T \colon \DO{T} \subseteq  \rightarrow X$ linear operator. We define
	\begin{itemize}
		\item We call the following set the \textbf{resolvent set}:
			$$\operatorname{res}(T) \coloneqq \rho(T) \coloneqq \big\{ \lambda \in \C ~|~\lambda \mathds{1} - T \text{ is bijective with bounded inverse} \big\}. $$
		\item The set $\operatorname{spec}(T) \coloneqq \sigma(T) \coloneqq \C \setminus \rho(T)$ is called \textbf{spectrum}.
		\item The set $\operatorname{spec}_p(T) \coloneqq \sigma_p(T) \coloneqq \big\{ \text{Eigenvalues of } T \big\}$ is the \textbf{point spectrum}.
		\item The following set is called the \textbf{continuous spectrum}: $ ~  \operatorname{spec}_c(T) \coloneqq \sigma_c(T)$
			$$ \sigma_c(T) \coloneqq \big\{ \lambda \in \C ~|~\lambda \mathds{1} - T \text{ is inj., but not surj., } \operatorname{range}(\lambda \mathds{1} - T) \text{ is dense in } X \big\} $$
		\item The following set is called the \textbf{residual spectrum}: $ ~ \operatorname{spec}_{res}(T) \coloneqq \sigma_{res}(T) $
				$$ \sigma_{res}(T) \coloneqq \big\{ \lambda \in \C ~|~\lambda \mathds{1} - T \text{ is inj., but not surj., } \operatorname{range}(\lambda \mathds{1} - T) \text{ is not dense in } X \big\} $$
		\item The \textbf{resolvent function}: $R_T \colon \rho(T) \rightarrow L(X, X) \eqqcolon L(X)$
			$$ \lambda \mapsto R_T(\lambda) \coloneqq R(\lambda, T) \coloneqq \left( \lambda \mathds{1} - T \right)^{-1} $$
	\end{itemize}
\end{definition}

\begin{remarks} ~\
	\begin{itemize}
		\item $\dim(X) < \infty: \sigma(T) = \sigma_p(T)$
		\item $\sigma(T) = \sigma_p(T) \overset{.}{\cup} \sigma_c(T) \overset{.}{\cup} \sigma_{res}(T)$
	\end{itemize}	
\end{remarks} ~\\

\begin{theorem}
	If $\rho(T) \neq \emptyset$ then $T$ is closed.
	
	\begin{proof}
		$\lambda \in \rho(T)$ then $\operatorname{graph}(R(\lambda, T))$ is closed (by the \hyperref[thm:a.cgt]{closed graph theorem}). For $x \in \DO{T}, y \in X$ with $R(\lambda, T) y = x$:
		$$ \| x \|_{\lambda \mathds{1} - T} = \| (\lambda - T) x \|_X + \| x \|_X = \| y \|_X + \| R(\lambda, T) y \|_X = \| y \|_{R(\lambda, T)}. $$
		Therefore, $\operatorname{graph}(\lambda \mathds{1} - T)$ and $\operatorname{graph}(R(\lambda, T))$ are isometric, and so $\lambda \mathds{1} - T$ is closed 
		$$ \Rightarrow T \text{ is closed} $$
	\end{proof}	
\end{theorem}

\begin{theorem}
	For a closed operator $T$ one has the equivalence
	$$ \lambda \in \rho(T) \iff \begin{cases} \operatorname{kern}(\lambda \mathds{1} - T) = 0, & \text{\enquote{inj.}} \\ \operatorname{range}(\lambda \mathds{1} - T) = X, & \text{\enquote{surj.}}\end{cases} $$
	
	\begin{proof} ~\
		\begin{description}
			\item[\hspace{0.25cm}\enquote{$\Rightarrow$}] By definition.
			\item[\hspace{0.25cm}\enquote{$\Leftarrow$}] Let $\lambda \in C$ with $\operatorname{kern}(\lambda \mathds{1} - T) = 0$, $\operatorname{range}(\lambda \mathds{1} - T) = X$. Then the inverse
			$$ \left( \lambda \mathds{1} - T \right)^{-1} \colon X \rightarrow X $$
			is defined everywhere and has a closed graph (as $\lambda \mathds{1} - T$ is closed), see proof of Theorem IV. 1. By the \hyperref[thm:acgt]{closed graph theorem} $\left( \lambda \mathds{1} - T \right)^{-1}$ is bounded, i.e. $\lambda \in \rho(T)$. 
		\end{description}
	\end{proof}
\end{theorem}

\begin{theorem}[Properties of the resolvent] ~\
	\begin{enumerate}[label=(\roman*\upshape)]
		\item For $\lambda_0 \in \rho(T)$, $\lambda \in \C$ with $\left| \lambda_0 - \lambda \right| < \frac{1}{\left\| R(\lambda_0, T) \right\|_{L(X)}}$, we have
			$$ R(\lambda, T) = \sum_{n=0}^\infty R(\lambda_0, T)^{n+1} \left( \lambda_0 - \lambda \right)^n $$
			and $\lambda \in \rho(T)$, i.e. $R(\cdot, T)$ is locally holomorphic, $\rho(T)$ is open, $\sigma(T)$ is closed.
		\item Resolvent equation
			$$ R(\lambda, T) - R(\mu, T) = \left( \mu - \lambda \right) R(\lambda, T) R(\mu, T) \quad \forall \lambda, \mu \in \rho(T) $$
			Note $\frac{1}{\lambda - T} - \frac{1}{\mu - T} = \frac{(\mu - T) - (\lambda - T)}{(\lambda - T)(\mu - T)} = \frac{(\mu - \lambda)}{(\lambda - T)(\mu - T)}$
		\item $R(\lambda, T) R(\mu, T) = R(\mu, T) R(\lambda, T) ~\forall \mu, \lambda \in \rho(T)$.
		\item $\frac{\partial}{\partial \lambda} R(\lambda, T) = - R(\lambda, T)^2 ~\forall \lambda \in \rho(T)$
		\item $T R(\lambda, T) = R(\lambda, T) T ~ \forall \lambda \in \rho(T)$
	\end{enumerate}
	
	\begin{proof} ~\
		\begin{enumerate}[label=(\roman*\upshape)]
			\item $\lambda - T = \left( \lambda - \lambda_0 \right) + \left( \lambda_0 - T \right) = \left( \lambda_0 - T \right) \left[ \mathds{1}_X - \frac{\left( \lambda_0 - \lambda \right)}{\left( \lambda - \lambda_0 \right)} R(\lambda_0, T) \right]$. ~\\
				Define $S \coloneqq \left( \lambda_0 - \lambda \right) R(\lambda_0, T) \in L(X)$ with $\|S \|_{L(X)} < 1$. By Neumann series:
				$$ R(\lambda, T) = \left( \mathds{1}_X - S \right)^{-1} R(\lambda_0, T) = \left( \sum_{n=0}^{\infty} S^n \right) R(\lambda_0, T) = \sum_{n=0}^\infty R(\lambda_0, T)^{n+1} \left( \lambda_0 - \lambda \right)^n $$
				and $\lambda \in \rho(T)$.
			\item $R(\lambda, T) - R(\mu, T) = R(\lambda, T) \left[ \mathds{1}_X - \left( \lambda - T \right) R(\mu, T) \right]$
				\begin{align*}
					 & = R(\lambda, T) \left[ \left( \mu - T \right) - \left( \mu - T \right) \right] R(\mu, T) ~\hspace{1.3cm} \\
					& = (\mu - \lambda) R(\lambda, T) R(\mu, T)
				\end{align*}
			\item $R(\lambda, T) R(\mu, T) \overset{(ii)}{=} \frac{R(\lambda, T) - R(\mu, T)}{\mu - \lambda} = \frac{R(\mu, T) - R(\lambda, T)}{\lambda - \mu} \overset{(ii)}{=} R(\mu, T) R(\lambda, T)$
			\item $\lim_{\mu \rightarrow \lambda} \frac{R(\mu; T) - R(\lambda, T)}{\mu - \lambda} \overset{(ii)}{=} \lim_{\mu \rightarrow \lambda} \frac{\left( \lambda - \mu \right) R(\mu, T) R(\lambda, T)}{\left( \mu - \lambda \right)} = - R(\lambda, T)^2$
			\item $T R(\lambda, T) - R(\lambda, T) T = \left( T - R(\lambda, T) T(\lambda - T) \right) R(\lambda, T) $
				$$ \hspace{1.5cm} = R(\lambda, T) \underbrace{\left( \left( \lambda - T \right) T - T \left( \lambda - T \right) \right)}_{= 0} R(\lambda, T) = 0$$
		\end{enumerate}
	\end{proof}
\end{theorem}

Next, we look at some basic examples:

\index{essential range}
\begin{definition}[essential range, dt. wesentlicher Bildbereich]
	Let $(\Omega, \mathcal{A}, \mu)$ be a measure space $f \colon \Omega \rightarrow \C$ measurable. Define the essential range of $f$ as:
	$$ \operatorname{essrange}(f) \coloneqq \big\{ \lambda \in \C ~|~\mu\left( \left\{ x \in \Omega \colon ~|\lambda - f(x)| < \epsilon \right\} \right) > 0 ~\forall \epsilon > 0 \big\} $$ 
\end{definition}

\begin{theorem}[Example: Spectrum of the multiplication operator]
	Let $f \in L_{loc}^\infty(\R^n)$, $M_d$ be the multiplication operator in $L^2(\R^n)$, i.e.
	$$ \DO{M_f} = \big\{ u \in L^2(\R^n) : fu \in L^2(\R^n) \}, ~M_f u = fu $$
	Then there holds:
	\begin{enumerate}[label=(\roman*\upshape)]
		\item $\sigma(M_f) = \operatorname{essrange}(f)$
		\item $\sigma_p(M_f) = \big\{ \lambda \in \C ~|~ \mathcal{L}^d \left( \{ x ~|~f(x) = \lambda \} \right) > 0 \big\}$
	\end{enumerate}
	
	\begin{proof} ~\
	  \begin{enumerate}[label=(\roman*\upshape)]
		\item \begin{description}
			\item[\enquote{$\subseteq$}] $\lambda \notin \operatorname{essrange}(f)$, i.e. $\exists c > 0$ s.t. $|\lambda - f(x)| \geq c$ for a.e. $x \in \R^n$.
				$$ \Rightarrow g \coloneqq \frac{1}{\lambda - f} \in L^\infty(\R^n) \Rightarrow gu \in L^2 ~\forall u \in L^2 $$
				$ \Rightarrow M_g \in L(L^2)$ with
				\begin{align*}
					M_g(\lambda - M_f) u(x) & = \frac{1}{\mu - f(x)} \left( \lambda - f(x) \right) u(x) \\
					&= u(x) = \left( \lambda - M_f \right) M_g u(x) \quad \forall u \in \DO{M_f} 
				\end{align*}
				i.e. $\lambda \in \rho(M_f)$ with $R(\lambda, M_f) = M_g$.
			\item[\enquote{$\supseteq$}] $A \in \operatorname{essrange}(f)$, we denote for any $m \in \N$:
					$$ \tilde{S_m} \coloneqq \big\{ x \in \R^n \colon ~ |\lambda - f(x)| < 2^{-m} \big\}. $$
				Choose $S_m \subseteq \tilde{S_m}$ s.t. $\mathcal{L}^d(S_m) \in (0, \infty)$; define
					$$ \phi_m(x) \coloneqq \begin{cases} 1, & x \in S_m \\ 0, & x \notin S_m \end{cases} $$
				Then
					\begin{align*}
						\left\| \left( \lambda - M_f \right) \phi_m \right\|^2_{L^2} & = \int_{S_m} \left| \lambda - f(x) \right|^2 \left| \phi_m(x) \right|^2 dx \\
							& \leq 2^{-2m} \| \phi_m\|_{L^2}^2 \quad \forall m \in \N
					\end{align*}
				$\Rightarrow$ If $\lambda \in \rho(M_f)$, but then
				$$ \left\| \left( \lambda - M_f \right)^{-1} \right\|_{L(L^2)} \geq \frac{\left\| \left( \lambda - M_f \right)^{-1} \left( \lambda - M_f \right) \phi_m \right\|_{L^2}}{\left\| \left( \lambda - M_f \right) \phi_m \right\|_{L^2}} \geq 2^{2m} \quad \forall m \in \N $$
				$\Rightarrow \lambda \in \sigma(M_f)$.
		   \end{description}
		\item \begin{description}
			\item[\enquote{$\Rightarrow$}] $\lambda \in \sigma_p(M_f) \iff \exists \phi \in L^2 \setminus \{ 0 \}$ s.t. $\left( \lambda - f(x) \right) \phi(x) = 0$ for a.e. $x$, i.e. $\phi = 0$ a.e. on $\{ y \in \R^d ~|~ f(y) \neq \lambda \}$, and therefore
				$$ \sigma_p(M_f) = \emptyset \iff L^d \left( \{ x ~|~f(x) = \lambda \} \right) = 0 $$
			\item[\enquote{$\Leftarrow$}] $\lambda \in \C$ s.t. $\mathcal{L}^d \left( \{ x ~|~f(x) = \lambda \} \right) > 0$. Take $S \subseteq \{ x ~|~f(x) = \lambda \}$ with $\mathcal{L}^d(s) \in (0, \infty)$, define
				$$ \phi_S(x) \coloneqq \begin{cases} 1, & x \in S \\ 0, & x \notin S \end{cases} $$
				Then:
					$$ f(x) \phi(x) = \lambda \phi(x) \text{ for a.e. } x \in \R^n $$
				$\Rightarrow \phi$ is Eigenfunction to the Eigenvalue $\lambda \Rightarrow \lambda \in \sigma_p(M_f)$.
		   \end{description}			
	  \end{enumerate}
	\end{proof}
\end{theorem}

\begin{example} ~\
	\begin{enumerate}
		\item ($\sigma(T) = \emptyset$) Setting: $H = L^2(0,1)$, $\DO{T} = \{ f \in W^{1,2}(0,1) ~|~ f(0) = 0 \}$, $Tf = f'$. ~\\
			Claim: $\sigma(T) = \emptyset, \sigma(T) = \C$.
			\begin{proof}
				For $g \in L^2(0,1)$, $\lambda \in \C$ the equation $(\lambda - T) f = g$ has the unique solution
				$$ f(x) = - \int_0^x e^{-\lambda(x-t)} g(t) dt, ~x \in (0, 1), f \in \DO{T} $$
				since $f'(x) = \lambda \int_0^x e^{\lambda (x-t)} g(t) dt - g(t) = \lambda f(x)-g(x)$. For $\tilde{f} \in \DO{T}$ with $(\lambda - T) \tilde{f} = g$ we get
				$$ \lambda(f - \tilde{f}) = \left(f - \tilde{f} \right)' \iff f - \tilde{f} = c e^{\lambda \cdot} \xRightarrow[]{\tilde{f}(0) = f(0)} c = 0 \Rightarrow f = \tilde{f} $$
				So $\lambda \in \rho(T)$, since
				\begin{align*}
					\| f \|_{L^2}^2 & = \int_0^1 \left| - \int_0^x e^{-\lambda(x-t)} g(t) dt \right|^2 dx \\
						& \leq \int_0^1 e^{2 |\lambda| x } \left| \int_0^1 |g(t)| dt \right|^2 dx \overset{Hölder}{\leq} \underbrace{\int_0^1 e^{2|\lambda|x} dx}_{< \infty} \| g \|_{L^2}^2
				\end{align*}
				$\Rightarrow \rho(T) = \C \Rightarrow \sigma(T) = \emptyset$.
			\end{proof}
		\item ($\sigma(T) = \C$) Setting: $H = L^2(0,1)$, $\DO{T} = W^{1,2}(0,1)$, $Tf = f'$. ~\\
			Claim: $\sigma(T) = \C, \sigma(T) = \emptyset$.
			\begin{proof}
				For $\lambda \in \C$: $\phi_\lambda(x) \coloneqq e^{\lambda x}$, $x \in \R$. $\phi_\lambda \in \DO{T}$ with $T \phi_\lambda = \lambda \phi_\lambda$.
					$$ \Rightarrow \sigma_p(T) = \C \Rightarrow \sigma(T) = \C, ~\rho(T) = \emptyset $$
			\end{proof}
	\end{enumerate}	
\end{example}

Following are going to be some basic facts on the spectra of self-adjoint operators.

\begin{lemma}
	Let $T$ be a closable operator in a Hilbert space $H$, $\lambda \in \C$, then
	\begin{enumerate}
		\item $\operatorname{kern} \left( \overline{\lambda} - T^* \right) = \operatorname{range} \left( \lambda - T \right)^\perp$
		\item $\overline{\operatorname{range}(\lambda - T)} = \operatorname{kern} (\overline{\lambda} - T^*)^\perp$
	\end{enumerate}
	
	\begin{proof} ~\
		\begin{enumerate}
			\item $\DO{T}$ is dense in $H$: 
				\begin{align*}
					f \in \operatorname{kern}(\overline{\lambda} - T^*) \iff & \langle g , \left( \overline{\lambda} - T^* \right) f \rangle_H = 0 \quad \forall g \in \DO{T} \\
						\iff & \langle g , T^* f \rangle_H = \lambda \langle g, f \rangle \quad \forall g \in \DO{T} \\
						\iff & \langle T g  f \rangle_H = \langle \lambda g, f \rangle \quad \forall g \in \DO{T} \\
						\iff & \langle \left( \lambda - T \right) g , f \rangle = 0 \quad \forall g \in \DO{T} \\
						\iff & f \in \operatorname{range}(\lambda - T)^\perp
				\end{align*}
			\item $\operatorname{kern} \left( \overline{\lambda} - T^* \right)^\perp = \left( \operatorname{range} \left( \lambda - T \right)^\perp \right)^\perp = \overline{ \operatorname{range} \left( \lambda - T \right)}$
		\end{enumerate}
	\end{proof}
\end{lemma}

\begin{theorem}[Spectrum of self-adjoint operator is real] \label[thm:iv.5]
	Let $T$ be a self-adjoint operator on $H$, then $\sigma(T) \subseteq \R$ and for $\lambda \in \C \setminus \R$:
		$$ \left\| R(\lambda, T) \right\|_{L(H)} \leq \frac{1}{\left| \operatorname{Im}(\lambda) \right|} $$	
		
	\begin{proof}
		Let $\lambda \in \C \setminus \R$, $u \in \DO{T}$:
		$$ \langle (\lambda - T) u, u \rangle = \lambda \langle u, u \rangle - \langle T u, u \rangle = \operatorname{Re}(\lambda) \langle u, u \rangle + i \operatorname{Im}(\lambda) \langle  u, u \rangle - \langle Tu, u \rangle, $$	
		$\langle T u, u \rangle = \langle u, T^* u \rangle \overset{T ~ s.a.}{=} \overline{\langle u, Tu \rangle} \Rightarrow \langle Tu, u \rangle \in \R$
			$$ \Rightarrow \left| \operatorname{Im}(\lambda) \right| \| u \|_H^2 \leq \left| \langle \left( \lambda - T \right) u, u \rangle \right| \overset{C.S.}{\leq} \left\| \left( \lambda - T \right) u \right\|_H \left\| u \right\|_H  $$
			\begin{align}
				\Rightarrow \left| \operatorname{Im}(\lambda) \right| \leq \left\| \left( \lambda - T \right) u \right\|_H \tag*{$(*)$},
			\end{align} 
			$\operatorname{kern}(\lambda - T) = \{ 0 \}$ and $\operatorname{range}(\lambda - T)$ is closed, since for $u \in \overline{\operatorname{range}(\lambda - T)}$, $(u_n)_n \subseteq \DO{T}$ s.t. $(\lambda - T) u_n \rightarrow u$ in $H$.
			$$ \Rightarrow \left( (\lambda - T) u_n \right)_n \text{ Cauchy in } H \xRightarrow[]{(*)} (u_n)_n \text{ Cauchy in } H $$
			$$ \xRightarrow[complete]{H} \exists u \in H \text{ s.t. } u_n \rightarrow u \text{ in } H \xRightarrow[closed]{\lambda - T} u \in \DO{T} with \left( \lambda - T \right) u = v. $$
			$$ \Rightarrow v \in \operatorname{range} \left( \lambda - T \right) $$
			$$ \xRightarrow[]{Lem. IV.1(ii)} \operatorname{range}(\lambda - T) = \{ 0 \}^\perp = H $$
			$\Rightarrow \lambda \in \rho(T), R(\lambda, T) \in L(H)$ with
			$$ \| R(\lambda, T) \|_{L(H)} = \sup_{u \neq 0} \frac{\| R(\lambda, T) u \|_H}{\| \left( \lambda - T \right) R(\lambda, T) u \|_H} \overset{(*)}{\leq} \sup_{u \neq 0} \frac{\| R(\lambda, T) u \|}{\left| \operatorname{Im}(\lambda) \right| \| R(\lambda, T) u \|} = \frac{1}{\left| \operatorname{Im}(\lambda) \right|} $$
	\end{proof}
\end{theorem}

\begin{lemma}[Spectrum of bounded operators]
	Let $T \in L(X)$ then:
		$$ \emptyset \neq \sigma(T) \subseteq \{ \lambda \in \C : |\lambda| \leq \| T \|_{L(X)} \}, $$
	i.e. $\sigma(T)$ is compact.
	
	\begin{proof}
		See a FA course.
	\end{proof}
\end{lemma}

\begin{theorem}[Location of spectrum of self-adjoint bounded operators]
	Let $T \in L(H)$ be self-adjoint. Denote
	$$ m \coloneqq m(T) \coloneqq \inf_{u \neq 0} \frac{\langle Tu, u \rangle}{\langle u, u \rangle}, ~ M \coloneqq M(T) \coloneqq \sup_{u \neq 0} \frac{\langle Tu, u \rangle}{\langle u, u \rangle} $$
	Then $\sigma(T) \subseteq [m, M], \{ m, M \} \subseteq \sigma(T)$.
	
	\begin{proof}
		By \hyperref[thm:iv.5]{Theorem IV.5}: $\sigma(T) \subseteq \R$. For $\lambda \in (M, \infty)$:
		$$ \left| \langle \left( \lambda - T \right) u, u \rangle \right| = \lambda \langle u, u \rangle - \langle T u, u \rangle \geq \left( \lambda - M \right) \langle u, u \rangle $$
	$\Rightarrow \left( \lambda - T \right)^{-1} \in L(H)$ by Lax-Milgram Theorem, $\lambda \in \rho(T)$. For $\lambda \in (-\infty, m)$: same way 
	$$\Rightarrow \sigma(T) \subseteq [m, M]. $$
		$M \in \sigma(T)$: obtain $(u, v) \mapsto \langle (M - T) u, v \rangle_H$ is an inner product. For $u,v \in H$ by C.S.-inequality
		$$ \left| \langle \left( M - T \right) u, v \rangle \right|^2 \leq \langle \left( M - T \right) u, z \rangle \langle \left(M - T \right)v, v \rangle $$
		$$ \Rightarrow \| \left( M - T \right) u \|_H^2 = \left| \sup_{\| v \| \leq 1} \langle \left(M-T\right)u,u\rangle \right| \langle \left(M - T\right)u, u \rangle = \left\| M - T \right\|_{L(H)} \langle \left(M - T \right) u, u \rangle $$
		By assumption construct $(u_n)_n \subseteq H$ s.t. $\| u_n \| = 1$.
		$$ \langle T u_n, u_n \rangle \rightarrow M $$
		$\Rightarrow \left( M - T \right) u_n \rightarrow 0 \Rightarrow M \notin \rho(T) \Rightarrow M \in \sigma(T)$. For $m$ in the same way.
	\end{proof}
\end{theorem}

\begin{lemma}
	$T = T^* \in L(H, H$ and $\operatorname{spec}(T) = \{ 0 \} \Rightarrow 0$.
	
	\begin{proof}
		Theorem IV.6 $\Rightarrow m = M = 0 \Rightarrow \langle Tx, x \rangle = 0 ~ \forall x \in H$
		$$ \xRightarrow[Forumla]{{\color{red}What?}} \langle Tx, y \rangle = 0 ~\forall x, y \in H \Rightarrow T = 0 $$ % todo pouldecation formula? What  is written here
	\end{proof}	
\end{lemma}

\begin{theorem}
	The spectrum of a self-adjoint operator on a Hilbert space is a non-empty closed subset of $\R$.
	
	\begin{proof}
		Assume $\operatorname{spec}(T) = \emptyset \Rightarrow T^{-1} \in L(H, H)$. Let $\lambda \in \C \setminus \{ 0 \}$.
		$$ L_{\lambda} \coloneqq - \frac{T}{\lambda} \left( ZT - \frac{1}{\lambda} \right)^{-1} = - \frac{1}{\lambda} - \frac{1}{\lambda^2} \left( T - \frac{1}{\lambda} \right)^{-1} \in L(H, H), ~ L_{\lambda}^{-1} = \left( T^{-1} - \lambda \right) $$
		$\Rightarrow \lambda \in \rho(T^{-1}) \Rightarrow \operatorname{spec}(T^{-1}) = \{ 0 \}$. $T^{-1}$ is self-adjoint by Theorem I.5 $\Rightarrow T^{-1} = 0$, which yields a contradiction.
	\end{proof}
\end{theorem}

\chapter{The Spectral Theorem}

Let $T$ be a self-adjoint operator on a Hilbert space $H$.

\begin{statements} ~\
		\begin{enumerate}[label=\arabic*\upshape)]
			\item $T$ is unitarily equivalent to a multiplication operator on a suitable $L^2$-Space (Finite dimension: every self-adjoint $n \times n$-matrix is unitarily equivalent to a diagonal matrix, i.e. a multiplication operator on $L^2\left( \{1, \dotsc, n\} \right)$.
			\item There exists a functional calculus for $T$, i.e. for all bounded Borell-functions $f$, $f(T)$ is defined and $f \mapsto f(T)$ is a homeomorphismus of the algebra of Borell-functions
				$$ f \colon X \rightarrow Y, ~ X, Y \text{\color{red} cup?}~ VS \text{ s.t. } f^{-1}(U) \in B(X) ~\forall U \text{ open} $$ % todo unreadable
				$$ \left( \text{Finite dimensions: } A = U \begin{pmatrix} \lambda_1 & ~ & ~ \\ ~ & \ddots & ~ \\ ~ & ~ & \lambda_n \end{pmatrix} U^{-1}, ~ f(A) \coloneqq U \begin{pmatrix} f(\lambda_1) & ~ & ~ \\ ~ & \ddots & ~ \\ ~ & ~ & f(\lambda_n) \end{pmatrix} U^{-1} \right) $$
			\item $T$ has a spectral representation. There exists a \text{\color{red} projection?}-valued measure $P_T$ on $\sigma(T)$, s.t. % todo unreadable
				$$ T = \int_{\sigma(T)} \lambda dP_T(\lambda) $$
				$$ \text{(Finite dimension: } A = \sum_{j=1}^n \lambda_j P_j, ~ P_j \text{ projection onto Eigenspace)} $$
				How to define $f(T)$? Polynomials: $f(x) = \sum_{j=0}^n c_j x^j$
					$$ f(T) = \sum_{j=0}^n c_j T^j, ~ D(f(t)) = ? $$
				Analytic $f(x) = \sum_{j=0}^\infty \frac{f^j(x_0)}{j!} (x - x_0)^j, ~ f(T) = \sum_{j=0}^\infty \frac{f^j(x_0)}{j!} (T - x_0)^j$ for $\| T - x_0 \| < \rho$
				$$ f(x) = (x - z)^{-1}, ~z \in \C \setminus \R \quad f(T) = (T - z)^{-1} = \operatorname{Res}_T(z) $$
		\end{enumerate}	
\end{statements}


\begin{unnamedtheorem*}[Cauchy integral Formula]
	Let $f$ be holomorph.
	$$ f(x_0) = \frac{i}{2 \pi} \int_\Gamma f(z) \left( x_0 - z \right)^{-1} dz = \frac{1}{2 \pi i} \int_{\Gamma} f(z) \left(z - x_0 \right)^{-1} dz $$
	$w = f(z) \left( z_0 - z \right)^{-1} dz$, $\C = \R^2$
	\begin{align*}
		z = x + i y, & \overline{z} = x - iy \\
		dz = dx + i dy, & d\overline{z} = dx - i dy
	\end{align*}
	$\partial_z = \frac{1}{2} \left( \partial_x - i \partial_y \right)$, $\partial_{\overline{z}} = \frac{1}{2} \left( \partial_x + i \partial_y \right)$
	\begin{align*}
		dw & = \frac{\partial w}{\partial z} \underbrace{dz \wedge dz}_{=0} + \frac{\partial w}{\partial \overline{z}} d\overline{z} \wedge dz \\
			& = \left( \partial_{\overline{z}} f(z) \right) \left(z_0 - z \right)^{-1} 2i dx \wedge dy
	\end{align*}
	$M \subset \C$ compact with smooth boundary $\Gamma$ and $B_\delta(z_0) \subseteq \overset{\circ}{M} \hookrightarrow M \setminus B_{\delta}(z_0)$
	$$ \int_{\partial \left( \Gamma \setminus B_{\delta(z_0)} \right)} w = \int_{\partial \Gamma} w - \int_{\partial B_{\delta}(z_0)} w = \int_{\Gamma \setminus B_{\delta}(z_0)} dw $$
	If $f$ is continuous at $z_0$, then
	\begin{align*}
		\lim_{\delta \rightarrow 0} \int_{\partial B_{\delta}(z_0)} w & = \lim_{\delta \rightarrow 0} \int_{\partial B_\delta(z_0)} \frac{f(z)}{z_0 - z} \\
			& = \lim_{\delta \rightarrow 0} \int_0^{2 \pi} \frac{f(z_0 + \delta e^{it}}{-\delta e^{it}} i \delta e^{it} dt \\
			& = - 2 \pi i f(z_0)
		\end{align*}
	$$ \lim_{\delta \rightarrow 0} \int_{\Gamma \setminus B_{\delta(z_0)}} \frac{\partial f}{\partial \overline{z}} \left( z_0 - z \right)^{-1} 2 i dx \wedge dy = 2i \int_{\Gamma} \frac{\partial f}{\partial \overline{z}} \left( z_0 - z \right)^{-1} dx \wedge dy $$
		$$ \Rightarrow f(z_0) = - \frac{1}{2 \pi i} \int_{\partial \Gamma} w + \frac{1}{\pi} \int_{\Gamma} \frac{\partial f}{\partial \overline{z}}(z) \left( z_0 - z \right)^{-1} dx \wedge dy $$
	If $f = 0$ on $\partial \Gamma$ then
		$$ f(z_0) = \frac{1}{\pi} \int_{\Gamma} \frac{\partial f}{\partial \overline{z}}(z) \left( z_0 - z \right)^{-1} dx \wedge dy $$
	($f \colon \C \rightarrow \C$ \text{\color{red}for} as $f \colon \R \rightarrow \C$). Now $\Gamma = \C$ % todo for?
	$$ f(T) = \frac{1}{\pi} \int_{\C} \frac{2f}{2 \overline{z}}(z) \left( T - z \right)^{-1} dx \wedge dy $$
	However, there 2 Problems:
	\begin{enumerate}[label=\arabic*\upshape)]
		\item maybe $f \colon \R \rightarrow \C$
		\item $\operatorname{spec}(T) \subset \C$
	\end{enumerate}
\end{unnamedtheorem*}

\index{almost analytic extension}
\begin{definition}
	Let $\tau \in C_c^\infty(\R)$ with $\operatorname{spec}(T(\tau)) \subset [-2, 2]$, $\tau |_{[-1, 1]} \equiv 1$. % todo is it really spec
		$$ \sigma(x, y) \coloneqq \tau \left( \frac{y}{\langle x \rangle} \right), ~\langle x \rangle \coloneqq \left( 1 + x^2 \right)^{\frac{1}{2}} $$
	For $f \in C^{\infty}(\R)$ and $n \geq 1$ we define the $n$-th \textbf{almost analytic extension} $\tilde{f}_n \colon \C \rightarrow \C$ by
		$$ \tilde{f}_n(z) \colon \sigma(x, y) \sum_{j=0}^n \frac{f^{j}(x)}{j!} \left(iy \right)^j $$
	\text{\color{red}Ruh}: % todo not readable
		\begin{align}
				\frac{\partial \tilde{f}}{\partial \overline{z}} (z) & = \frac{1}{2} \left( \partial_x - \frac{1}{i} \partial_y \right) \tilde{f}_n(x, y) \\
				& = \frac{1}{2} \left( \sum_{j=0}^{n} \frac{f^{(j+1)}(x)}{j!} \left(iy \right)^j - \sum_{j=1}^{n} \frac{f^{(j)}(x)}{(j-1)!} \left( i y \right)^{i-1} \right) \sigma(x,y)  + \frac{1}{2} \sum \dotsc \left( \partial_x \sigma + i \partial_y \sigma \right) \\
				& = \frac{1}{2} \frac{f^{(n+1)}(x)}{n!}	\left( iy \right)^n \sigma(x,y) + \dotsc
		\end{align}
		$\sigma \equiv 1$ on a strip of size $\langle x \rangle$ around $\R$
		$$ \Rightarrow \left| \frac{\partial \tilde{f}_n(z)}{\partial \overline{z}} \right| = \mathcal{O} \left( \left| y \right|^n \right) \text{ for } y \rightarrow 0 $$
\end{definition}

\begin{definition}
	Let $T = T^*$ be a map which associates to every element $f \colon \R \rightarrow \C$ of a subalgebra $\mathcal{E}$ of the Borelfunctions $\mathcal{B}(\R)$ an operator $f(T) \in L(H, H)$ is called functional calculus for $T$ if
	\begin{enumerate}[label=\roman*\upshape)]
		\item $f \mapsto f(T)$ is an algebra homomorphism
			$$ \left( f + \alpha g \right)(T) = f(T) + \alpha g(T), ~ \left(f \cdot g \right) (T) = f(T) \cdot g(T) \quad \forall f,g \in \mathcal{E} $$
		\item $f(T)^* = \overline{f}(T)$
		\item $\| f(T) \| \leq \| f \|_{L^\infty}$
		\item For $z \in \C \setminus \R$ and $r_z(x) = (x - z)^{-1}$ is 
			$$ r_z(T) = \operatorname{Res}_T(z) $$
		\item If $f \in C_c^\infty(\R)$ \text{\color{red}vandities} on $\operatorname{spec}(T)$, i.e.
			$$ \operatorname{sp} T(f) \cap \operatorname{spec}(T) = \emptyset, \text{\color{red} then } f(t) = 0  $$ % todo was ist sp? and then?
	\end{enumerate}
\end{definition}

\begin{definition}
	For $\beta \in \R$ let 
	$$ S^\beta \coloneqq \big\{ f \in C^\infty(\R): ~\forall n \in \N_0 ~\exists c_n < \infty \text{ s.t. } \left| f^{(n)}(x) \right| \leq c_n \langle x \rangle^{\beta - n } ~\forall x \in \R \big\} $$
	$\mathcal{A} \coloneqq \bigcup_{\beta < 0} S^\beta$
		$$ \| f \|_n \coloneqq \sum_{j=0}^{n} \int_{-\infty}^\infty \underbrace{\left| f^{(j)}(x) \right| \langle x \rangle^{j-1}}_{\sim \langle x \rangle^{\beta - 1} $$
		well-defined
\end{definition}


\appendix

\chapter*{Addendum}

\begin{theorem*}[Riesz' representation theorem, FA 17.2]
Let $H$ be a Hilbert space, and let $H'$ denote its dual space, consisting of all continuous linear functionals from $H$ into the field ($\C$ or $\R$). For every element of $x' \in X'$there exists a unique $x \in X$ such that
	$$ x'(y) = \langle y, x \rangle,  $$
	for all $y \in X$, and $\|x'\|_{X'} = \|x\|_X$.
\end{theorem*}

\begin{theorem*}[Closed graph theorem, FA 12.6] \label[thm:acgt]
	 If $X$ and $Y$ are Banach spaces, and $T \colon X \rightarrow Y$ is a linear operator, then $T$ is continuous if and only if its graph is closed in $X \times Y$, with respect to the product topology.
\end{theorem*}

\begin{definition*}[Isometric]
	Let $X$ and $Y$ be metric spaces with metrics $d_X$ and $d_Y$. A map $f \colon X \rightarrow Y$ is called an isometry or distance preserving if for any $a,b \in X$ one has
		$$ d_Y \left( f(a), f(b) \right) = d_X\left(a, b\right). $$

	$X$ and $Y$ are called isometric if there is a bijective isometry from $X$ to $Y$. 
\end{definition*}

% Index									
\renewcommand{\indexname}{Stichwortverzeichnis}
\printindex


\end{document}