\documentclass[12pt]{extreport} % Schriftgröße: 8pt, 9pt, 10pt, 11pt, 12pt, 14pt, 17pt oder 20pt

%% Packages
\usepackage{scrextend}
\usepackage{amssymb}
\usepackage{amsthm}
\usepackage{booktabs}
\usepackage{chngcntr}
\usepackage{cmap}
\usepackage{color}
\usepackage{csquotes}
\usepackage{enumitem}
\usepackage{float}
\usepackage{bbding}
\usepackage{hyperref}
\usepackage{ulem}
\usepackage{lmodern}
\usepackage{makeidx}
\usepackage{mathtools}
\usepackage{xpatch}
\usepackage{pgfplots}
\pgfplotsset{compat=1.7}
\usetikzlibrary{calc}	
\usetikzlibrary{matrix}	

% Language Setup (English)
\usepackage[utf8]{inputenc} 
\usepackage[T1]{fontenc} 
\usepackage[english]{babel}

% Options
\makeatletter%%  
  % Linkfarbe, {0,0.35,0.35} für Türkis, {0,0,0} für Schwarz, {1,0,0} für Rot, {0,0,0.85} für Blau
  \definecolor{linkcolor}{rgb}{0,0.35,0.35}
  % Zeilenabstand für bessere Leserlichkeit
  \def\mystretch{1.2} 
  % Publisher definieren
  \newcommand\publishers[1]{\newcommand\@publishers{#1}} 
  % Enumerate im 1. Level: \alph für a), b), ...
  \renewcommand{\labelenumi}{\alph{enumi})} 
  % Enumerate im 2. Level: \roman für (i), (ii), ...
  \renewcommand{\labelenumii}{(\roman{enumii})}
  % Zeileneinrückung am Anfang des Absatzes
  \setlength{\parindent}{0pt} 
  % Für das Proof-Environment: 'Beweis:' anstatt 'Beweis.'
  \xpatchcmd{\proof}{\@addpunct{.}}{\@addpunct{:}}{}{} 
  % Nummerierung der Bilder, z.B.: Abbildung 4.1
  \@ifundefined{thechapter}{}{\def\thefigure{\thechapter.\arabic{figure}}} 
  % Chapter-Nummerierung beginnen bei (0):
  \setcounter{chapter}{0}
  % Chapter-Nummerierung
  \renewcommand\thechapter{\Roman{chapter}}
\makeatother%

% Meta Setup 
\title{Spectraltheory}
\author{Prof. Dr. Tobias Lamm}
\date{Sommersemester 2017}
\publishers{Karlsruher Institut für Technologie}

%% Math. Definitiones
\newcommand{\C}{\mathbb{C}}
\newcommand{\N}{\mathbb{N}}
\newcommand{\Q}{\mathbb{Q}}
\newcommand{\R}{\mathbb{R}}
\newcommand{\Z}{\mathbb{Z}}
\newcommand{\DO}[1]{\mathcal{D}\left( {#1} \right)}
\newcommand{\RO}[1]{\mathcal{R}\left( {#1} \right)}

\newtheoremstyle{named}{}{}{\normalfont}{}{\bfseries}{:}{0.25em}{#2 \thmnote{#3}}
\newtheoremstyle{nnamed}{}{}{\normalfont}{}{\bfseries}{:}{0.25em}{\thmnote{#3}}
\newtheoremstyle{itshape}{}{}{\itshape}{}{\bfseries}{:}{ }{}
\newtheoremstyle{normal}{}{}{\normalfont}{}{\bfseries}{:}{ }{}
\renewcommand*{\qed}{\hfill\ensuremath{\square}}

\theoremstyle{named}
\newtheorem{unnamedtheorem}{Theorem} \counterwithin{unnamedtheorem}{chapter}
\theoremstyle{nnamed}
\newtheorem*{unnamedtheorem*}{Theorem} 

\theoremstyle{itshape}
\newtheorem{definition}{Definition}  \counterwithin{definition}{chapter}
\newtheorem{theorem}{Theorem}  \counterwithin{theorem}{chapter}
\newtheorem{lemma}{Lemma}  \counterwithin{lemma}{chapter}

\theoremstyle{normal}
\newtheorem*{theorem*}{Theorem}
\newtheorem*{recall}{Recall}
\newtheorem*{example}{Example}
\newtheorem*{remark}{Remark}
\newtheorem*{satz}{Satz}
\newtheorem*{bemerkung}{Bemerkung}

%% Template
\makeatletter%
\DeclareUnicodeCharacter{00A0}{ } \pgfplotsset{compat=1.7} \hypersetup{colorlinks,breaklinks, urlcolor=linkcolor, linkcolor=linkcolor, pdftitle=\@title, pdfauthor=\@author, pdfsubject=\@title, pdfcreator=\@publishers}\DeclareOption*{\PassOptionsToClass{\CurrentOption}{report}} \ProcessOptions \def\baselinestretch{\mystretch} \setlength{\oddsidemargin}{0.125in} \setlength{\evensidemargin}{0.125in} \setlength{\topmargin}{0.5in} \setlength{\textwidth}{6.25in} \setlength{\textheight}{8in} \addtolength{\topmargin}{-\headheight} \addtolength{\topmargin}{-\headsep} \def\pulldownheader{ \addtolength{\topmargin}{\headheight} \addtolength{\topmargin}{\headsep} \addtolength{\textheight}{-\headheight} \addtolength{\textheight}{-\headsep} } \def\pullupfooter{ \addtolength{\textheight}{-\footskip} } \def\ps@headings{\let\@mkboth\markboth \def\@oddfoot{} \def\@evenfoot{} \def\@oddhead{\hbox {}\sl \rightmark \hfil \rm\thepage} \def\chaptermark##1{\markright {\uppercase{\ifnum \c@secnumdepth >\m@ne \@chapapp\ \thechapter. \ \fi ##1}}} \pulldownheader } \def\ps@myheadings{\let\@mkboth\@gobbletwo \def\@oddfoot{} \def\@evenfoot{} \def\sectionmark##1{} \def\subsectionmark##1{}  \def\@evenhead{\rm \thepage\hfil\sl\leftmark\hbox {}} \def\@oddhead{\hbox{}\sl\rightmark \hfil \rm\thepage} \pulldownheader }	\def\chapter{\cleardoublepage  \thispagestyle{plain} \global\@topnum\z@ \@afterindentfalse \secdef\@chapter\@schapter} \def\@makeschapterhead#1{ {\parindent \z@ \raggedright \normalfont \interlinepenalty\@M \Huge \bfseries  #1\par\nobreak \vskip 40\p@ }} \newcommand{\indexsection}{chapter} \patchcmd{\@makechapterhead}{\vspace*{50\p@}}{}{}{}\def\Xint#1{\mathchoice
    {\XXint\displaystyle\textstyle{#1}} {\XXint\textstyle\scriptstyle{#1}} {\XXint\scriptstyle\scriptscriptstyle{#1}} {\XXint\scriptscriptstyle\scriptscriptstyle{#1}} \!\int} \def\XXint#1#2#3{{\setbox0=\hbox{$#1{#2#3}{\int}$} \vcenter{\hbox{$#2#3$}}\kern-.5\wd0}} \def\dashint{\Xint-} \def\Yint#1{\mathchoice {\YYint\displaystyle\textstyle{#1}} {\YYYint\textstyle\scriptscriptstyle{#1}} {}{} \!\int} \def\YYint#1#2#3{{\setbox0=\hbox{$#1{#2#3}{\int}$} \lower1ex\hbox{$#2#3$}\kern-.46\wd0}} \def\YYYint#1#2#3{{\setbox0=\hbox{$#1{#2#3}{\int}$}  \lower0.35ex\hbox{$#2#3$}\kern-.48\wd0}} \def\lowdashint{\Yint-} \def\Zint#1{\mathchoice {\ZZint\displaystyle\textstyle{#1}}{\ZZZint\textstyle\scriptscriptstyle{#1}} {}{} \!\int} \def\ZZint#1#2#3{{\setbox0=\hbox{$#1{#2#3}{\int}$}\raise1.15ex\hbox{$#2#3$}\kern-.57\wd0}} \def\ZZZint#1#2#3{{\setbox0=\hbox{$#1{#2#3}{\int}$} \raise0.85ex\hbox{$#2#3$}\kern-.53\wd0}} \def\highdashint{\Zint-} \DeclareRobustCommand*{\onlyattoc}[1]{} \newcommand*{\activateonlyattoc}{ \DeclareRobustCommand*{\onlyattoc}[1]{##1} } \AtBeginDocument{\addtocontents{toc} {\protect\activateonlyattoc}} 
	% Titlepage
	\def\maketitle{ \begin{titlepage} 
			~\vspace{3cm} 
		\begin{center} {\Huge \@title} \end{center} 
	 		\vspace*{1cm} 
	 	\begin{center} {\large \@author} \end{center} 
	 	\vspace*{-0.5cm}
	 	\begin{center} \@date \end{center} 
	 		\vspace*{7cm} 
	 	\begin{center} \@publishers \end{center} 
	 		\vfill 
	\end{titlepage} }
\makeatother%

% Create Index
\makeindex 

\begin{document}

\pagenumbering{Alph}
\begin{titlepage}
	\maketitle
	\thispagestyle{empty}
\end{titlepage}

% Table of Contents
\tableofcontents
\thispagestyle{empty}

% Lecture Notes - Start 			
\pagenumbering{arabic}

\chapter{Unbounded, adjoint and self-adjoint operators}


Let $H$ be a separable Hilbert space, i.e. a real or complex inner product space that is also a complete metric space with respect to the distance function induced by the inner product $\langle \cdot, \cdot \rangle$ on $H$. ~\bigskip

\begin{recall} \index{inner product}
	A mapping $\langle \cdot, \cdot \rangle \colon H \times H \rightarrow \C$
	is called an \textbf{inner product}, if for all $x, y \in H$, $\lambda \in \C$ holds:
	\begin{description}
	 	\item[$\hspace{0.5cm} (S1) \hspace{0.1cm} $] $\langle x_{1} + x_{2}, y \rangle = \langle x_{1}, y \rangle + \langle x_{2}, y \rangle$, $\langle x, y_{1} + y_{2} \rangle = \langle x, y_{1} \rangle + \langle x, y_{2} \rangle$
	 	\item[$\hspace{0.5cm} (S2) \hspace{0.1cm} $] $\langle  \lambda x, y \rangle = \lambda \langle x, y \rangle,$ $\langle x, \lambda y \rangle = \overline{\lambda} \langle x, y \rangle$
	 	\item[$\hspace{0.5cm} (S3) \hspace{0.1cm} $] $\langle  x, y \rangle = \overline{\langle y, x \rangle }$
	 	\item[$\hspace{0.5cm} (S4) \hspace{0.1cm} $] $\langle  x, x \rangle \geq 0, \quad \langle x, x \rangle = 0 \iff x = 0$
	\end{description}	
\end{recall} ~\smallskip

\index{domain} \index{range}
A linear operator $T$ in $H$ is a linear map 
	$$ u \mapsto Tu $$
defined on a subspace $\DO{T}$ of $H$, and we call $\DO{T}$ the \textbf{domain} of $T$. For $T \colon \DO{T} \rightarrow H$ we denote the \textbf{range} of $T$ with
$$ \RO{T} \coloneqq \operatorname{Image}\left( T \right). $$
\newpage
\index{bounded} \index{Operator!bounded}
We say that $T$ is \textbf{bounded} if it is continuous from $\DO{T}$ into $H$, with respect to the topology induced by $H$. We recall that if $\DO{T} = H$ holds, boundedness of a linear operator is equivalent to continuity in $0$, boundedness of $T(U_{(X,\|\cdot\|)})$ in $Y$ and that there $\exists c < \infty$ such that $\| T x \| \leq c \| x \|$, for proof see theorem 3.3 in the \href{https://github.com/MBelica/Funktionalanalysis-WS2015}{functional analysis} course. ~\bigskip

\index{dense}
 From now on, if $\DO{T} \neq H$ we will assume that $\DO{T}$ is \textbf{dense} in $H$, i.e. $\overline{\DO{T}} = H$. If in this case $T$ would be bounded then $T$ has a unique continuous extension to all of $H$, for proof see proposition 5.10 in the \href{https://github.com/MBelica/Funktionalanalysis-WS2015}{functional analysis} course.  As this simplifies many considerations some of the following theorems would be trivial, and hence, we won't focus on bounded operators during this lecture.

\begin{recall} \index{closed} \index{Operator!closed} \index{Graph}
	An operator is called \textbf{closed} if the graph 
	$$ G(T) \coloneqq \big\{ (x, y) \in H \times H ~ \big| ~ x \in \DO{T}, y = Tx \big\} $$
	is closed in $H \times H$.
\end{recall}

\index{closed} \index{Operator!closed}
\begin{definition} \label{i.1:def}
	Let $T \colon \DO{T} \rightarrow H$ be a (linear) operator with $\DO{T}$ dense in $H$. Then $T$ is called \textbf{closed} if for all 
		$$ u_n \in \DO{T}, ~ u_n \rightarrow u \in H \text{ and } T u_n \rightarrow v \in H $$
	follows that
		$$ u \in \DO{T}, ~ v = T u $$
	holds.
\end{definition}

\begin{example} ~\
	\begin{enumerate}
		\item Let $H = L^2(\R^n)$, then $\DO{T_{0}} = C_c^\infty(\R^n)$ is dense in $H$. Define the operator
			$$ T_0 = - \Delta,  $$
			and take $u \in W^{2,2}(\R^n) \setminus C_c^{\infty}(\R^n)$, s.t $u \in L^2(\R^n)$. Due to the density:
			$$ \exists ~(u_n)_{n \in \N} \subseteq C_c^{\infty}(\R^n) \colon ~ u_n \rightarrow u \text{ in } W^{2,2}(\R^n). $$
			As a result, $\left( u_n, -\Delta u_n \right) \in G(T_0)$ converges in $L^2 \times L^2$ to $(u, -\Delta u) \notin G(T_0)$.
		\item Let $H = L^2(\R^n)$, and set $\DO{T_1} = W^{2,2}(\R^n) \subseteq H$. Define the operator
			$$ T_1 = - \Delta. $$
		 For $(u_n)_{n \in \N} \subseteq \DO{T_1}$ with 
			$$ u_n \rightarrow u \in H \text{ and } \left( - \Delta u_n \right) \rightarrow v \in L^2 $$
			follows that $- \Delta u = v \in L^2(\R^n)$ weakly, i.e. for all $\varphi \in C_c^{\infty}(\R^n)$:
			$$ \int_{\R^n} v \varphi \longleftarrow \int_{\R^n} \left( - \Delta u_n \right) \varphi = \int_{\R^n} u_n \left( - \Delta \varphi \right) \longrightarrow \int_{\R^n} u \left( - \Delta \varphi \right). $$
			$\xRightarrow[]{PDE} u \in W^{2,2}(\R^n) = \DO{T_1} \Rightarrow T_1$ is closed.
	\end{enumerate}
\end{example}

\index{closable} \index{Operator!closable}
\begin{definition}
	An operator $T$ is called \textbf{closable} $\iff \overline{G(T)}$ is a graph.
\end{definition}

\index{closure} \index{Operator!closure of}
\begin{remark}
	We call $\overline{T}$ the \textbf{closure} of $T$, and in such case we have 
	$$ \DO{\overline{T}} \coloneqq \big\{ x \in H ~ \big| ~ \exists ~ y \colon (x, y) \in \overline{G(R)} \big\} $$	
	For any $x \in \DO{\overline{T}}$ the assumption that $\overline{G(T)}$ is a graph implies that $y$ is unique and hence
	$$ \Rightarrow G(\overline{T}) = \overline{G(T)}, ~\overline{T} x \coloneqq y $$
	Equivalently, $\DO{\overline{T}}$ is the set of all $x \in H$ such that there exists a sequence $x_n \in \DO{T}$ with $x_n \rightarrow x$ in $H$ and $T x_n$ is a cauchy sequence. For such $x$ we define
	$$ \overline{T}x \coloneqq \lim_{n \rightarrow \infty} T x_n $$
\end{remark}

\begin{example}
	Let $T_0 \coloneqq - \Delta$, $\DO{T_0} = C_c^\infty(\R^n)$ is closable with $\overline{T_0} = T_1$.
	 
	\begin{proof}
		Let $u \in L^2(\R^n)$ such that there exists $(u_n)_{n \in \N} \subseteq C_c^\infty(\R^n)$ with $u_n \rightarrow u$ in $L^2$ and $-\Delta u_n \rightarrow u$ in $L^2$, as above: 
		$$ -\Delta u = v \in L^2 $$
		For a given $u$ the function $v$ is unique, and hence, $T_0$ is closable. Let $\overline{T_0}$ be the closure with domain $\DO{\overline{T_0}}$ and $u \in \DO{T_0}$
		$$ \Rightarrow \Delta u \in L^2 \Rightarrow u \in W^{2,2}(\R^n) = \DO{T_1} \Rightarrow \DO{\overline{T_0}} \subseteq \DO{T_1} $$
		but $C_c^{\infty}(\R^n)$ is dense in $\DO{T_1} = W_{2,2}(\R^n)$
		$$ \Rightarrow W^{2,2}(\R^n) \subseteq \DO{\overline{T_0}} \Rightarrow \DO{\overline{T_0}} = \DO{T_1} $$
		$\Rightarrow T_1 = \overline{T_0}$. 
	\end{proof}	
\end{example}

\begin{remark}
	Assume for a second in the example above that $W^{2,2} \not\subseteq \DO{T_0}$ holds:
		$$ \Rightarrow \exists u \in W^{2, 2} \setminus D \left( \overline{T_0} \right), ~\exists (u_n)_{n \in \N} \in C^{\infty}_c (\R^n): u_n \rightarrow u \text{ in } W^{2, 2}, $$
		using the same arguments as in the example above $\Rightarrow \overline{T_0}$ not closed!	
\end{remark}


\begin{recall}
	If $T \colon H \rightarrow H$ is bounded then $T^*$ is defined through
	$$ \langle u , T^* v \rangle = \langle T u , v \rangle, ~ \forall u, v \in H $$	
	$u \mapsto \langle T u, v \rangle$ defines a continuous linear map on $H$ ($\in H'$). Riesz' representation theorem then ensures the existence of $T^*$.
\end{recall}

\begin{definition}
	If $T$ is an unbounded operator on $H$ with dense domain we define
		$$ \DO{T^*} \coloneqq \big\{ v \in H \colon \DO{T} \ni u \mapsto \langle T u, v \rangle \text{ can be extended as a linear continuous form on } H \big\} $$
	Using Riesz' representation theorem $\exists! f \in H$:
		$$ \langle u, f \rangle = \langle T u, v \rangle, ~ \forall u \in \DO{T} $$
	then define $T^* v = f$, where the uniqueness follows from the density of $\DO{T}$ in $H$.
\end{definition}

\begin{remark}
	If $\DO{T} = H$ and $T$ is bounded then we recover the \enquote{old} adjoint.
\end{remark}

\begin{example}
	$T_0^* = T_1$, 
	\begin{align*}
		\DO{T_0^*} & = \big\{ v \in L^2(\R^n) \colon C_c^\infty(\R^n) \ni u \mapsto \langle - \Delta u, v \rangle \text{ extendable as a lin. continuous form on } L^2(\R^n) \big\} \\
			& = \big\{ v \in L^2(\R^n) \colon - \Delta v \in L^2(\R^n) \big\} = W^{2,2}(\R^n) = \DO{T_1}
	\end{align*}
	Damit ist
	$$ \langle T_1 u, v \rangle = \langle - \Delta u, v \rangle = \int v \left( - \Delta u \right) = \int \left( - \Delta v \right) u = \langle u, T_1 v \rangle $$
\end{example}

\begin{theorem} \label{I.1:thm}
	$T^*$ is a closed operator.
	
	\begin{proof}
		$v_n \in \DO{T^*}$ such that $v_n \rightarrow v$ in $H$ and $T^* v_n \rightarrow w^*$ in $H$ for $(v, w^*) \in H \times H$. ~\\
		For all $u \in \DO{T}$ we have
			$$ \langle Tu, v \rangle = \lim_{n \rightarrow \infty} \langle T u, v_n \rangle = \lim_{n \rightarrow \infty} \langle u, T^* v_n \rangle = \langle u , w^* \rangle $$
		($H \ni u \mapsto \langle u, w^* \rangle$ is continuous) $\Rightarrow v \in \DO{T^*}$ and $w^* = T^* v$ by definition.
	\end{proof}
\end{theorem}

\begin{theorem} \label{I.2:thm}
	Let $T$ be an operator in $H$ with domain $\DO{T}$. Then
		$$ G \left( T^* \right) = \left( V \left( \overline{G(T)} \right) \right)^{\perp} $$
	where $V \colon H \times H \rightarrow H \times H, V(x, y) = (y, -x)$ ($V^2 = - \mathbb{I}$).
	
	\begin{proof}
		Let $u \in \DO{T}, \left(v, w^* \right) \in H \times H$ 
			$$ \Rightarrow \langle V \left( u, Tu \right), \left( v, w^* \right) \rangle_{H \times H} = \langle T u, v \rangle - \langle u, w^* \rangle $$
		Considering the right-hand side it follows
		$$ \langle T u, v \rangle - \langle u, w^* \rangle = 0 ~ \forall u \in \DO{T} \iff v \in \DO{T^*} \text{ and } w^* = T^* v \iff \left( v, w^* \right) \in G\left( T^* \right), $$
		and considering the left-hand side:
		$$ \Rightarrow \langle V \left( u, Tu \right), \left( v, w^* \right) \rangle_{H \times H} = 0 ~ \forall u \in \DO{T} \iff \left( v, w^* \right) \in V \left( G(T) \right)^{\perp} $$
		In general: $U^{\perp} = \overline{U}^{\perp}$, and hence
		$$ \Rightarrow V \left( G(T) \right)^{\perp} = \left( \overline{V\left( G(T) \right)} \right)^{\perp} = \left( V \left( \overline{G(T)} \right) \right)^{\perp}. $$
	\end{proof}
\end{theorem}

\begin{theorem} \label{I.3:thm}
	Let $T$ be a closable operator. Then:
	\begin{enumerate}
		\item $\DO{T^*}$ is dense in $H$
		\item $T^{**} \coloneqq \left( T^* \right)^* = \overline{T}$
	\end{enumerate} 
	
	\begin{proof} ~\
		\begin{enumerate}
			\item Proof through contradiction: $D\left( T^* \right)$ not dense in $H \rightarrow \exists w \neq 0: \langle w, v \rangle = 0 ~\forall v \in \overline{\DO{T^*}}$
				\begin{align*}
					& \xRightarrow[]{~~~~} \langle \left( 0, w \right), \left( T^*v, -v \right) \rangle_{H \times H} = 0 ~\forall v \in \DO{T^*} \\
					& \xRightarrow[]{~~~~} \left( 0, w \right) \perp V \left( G(T^*) \right) \\
					& \xRightarrow[\ref{I.2:thm}]{\hyperref[I.2:thm]{Thm}} V \left( \overline{G(T)} \right) = G \left( T^* \right)^{\perp} \\
					& \xRightarrow[]{~~~~} V \left( G(T^*)^{\perp} \right) = \overline{G(T)}
				\end{align*}
				For any $M \subseteq H \times H$ we have $V \left( M^{\perp} \right) = V(M)^{\perp}$ since for $(u,v) \in V(M)^{\perp}$, $(x,y) \in M$
				$$ \langle V(u, v) , (x,y) \rangle_{H \times H} = - \langle (u, v ), V(x,y) \rangle_{H \times h} \Rightarrow V(u, v) \in M^{\perp} \Rightarrow (u, v) \in V\left( M^{\perp} \right) $$
				$\Longrightarrow V \left( G \left( T^* \right) \right)^{\perp} = \overline{G(T)} = G \left(\overline{T} \right) \Longrightarrow (0, w) \in G\left( \overline{T} \right) \Longrightarrow w = 0$
			\item $G \left( T^{**} \right) \overset{\hyperref[I.2:thm]{Thm}}{\underset{\ref{I.2:thm}}{=}} V \left( \overline{G \left( T^* \right)} \right)^{\perp} \overset{\hyperref[I.1:thm]{Thm}}{\underset{\ref{I.1:thm}}{=}} V \left( G \left( T^* \right) \right)^{\perp} \overset{(\perp)}{=} G \left( \overline{T} \right) \Longrightarrow \DO{T^{**}} = \DO{\overline{T}}, T^{**} = \overline{T}$
		\end{enumerate}
	\end{proof}
\end{theorem}

\begin{definition}
	We say $T \colon \DO{T} \rightarrow H$ is \textbf{symmetric} if and only if
	$$ \langle Tu, v \rangle = \langle u , Tv \rangle \quad \forall u, v \in \DO{T} $$
\end{definition}

\begin{example}
	$T_0 = - \Delta$, $\DO{T_0} = C_c^{\infty}(\R^n)$
	$$ \int_{\R^n} \left( - \Delta u \right) v = \int_{\R^n} u \left( - \Delta v \right) $$	
\end{example}

\begin{remark}
	If $T$ is symmetric $\Rightarrow \DO{T} \subseteq \DO{T^*}$ and
	$$ T u = T^* u \quad \forall u \in \DO{T} $$
	$\Rightarrow \left( T^*, \DO{T^*} \right)$ is an extension of $\left( T, \DO{T} \right)$.
\end{remark}

\begin{lemma} \label{I.1:lmm}
	A symmetric operator $T$ is closable.
	
	\begin{proof}
		It suffice to show that for $u_n \in \DO{T}$ with $u_n \rightarrow 0$ and $ T u_n \rightarrow x \in H$ we have $x = 0$
		$$ \langle x,  v \rangle \leftarrow \langle T u_n, v \rangle = \langle u_n , T v \rangle \rightarrow \langle 0, T u \rangle = 0 \quad \forall v \in \DO{T}  $$
		$\Rightarrow x = 0$.
	\end{proof}
\end{lemma}

\begin{remark}
	The proof actually shows that if $\DO{T^*}$ is dense in $H$, then $T$ is closable.	
\end{remark}

\begin{definition}
	We call an operator $T$ self-adjoint if 
		$$ T = T^* \text{ and }  Tu = T^* u \quad \forall u \in \DO{T},  $$ % todo wieso ist diese Forderung nötig?
	note that the first property implies that $\DO{T} = \DO{T^*}$.
\end{definition}

\begin{theorem} \label{thm:1.4}
	Every self-adjoint operator is closable.
	
	\begin{proof}
		\hyperref[I.1:lmm]{Lemma I.1}
	\end{proof}
\end{theorem}

\begin{theorem}
	Let $T$ be an invertible self-adjoint operator, then $T^{-1}$ is also self-adjoint.
	
	\begin{proof}
		For $T \colon \DO{T} \rightarrow \RO{T}$ consider
		\begin{description}
			\item[~\hspace{0.25em}~Step 1] $\RO{T}$ is dense in $H$. We have to show that $\RO{T}^\perp = \{ 0 \}$. ~\\
				Let $w \in H$ such that 
					$$\langle T u, w \rangle = 0 ~\forall u \in \DO{T}$$
				$\Longrightarrow w \in \DO{T^*}$ and $T^* w = 0 \xRightarrow[s.a.]{inj.} w = 0$.
			\item[~\hspace{0.25em}~Step 2] Let $w \colon H \times H \rightarrow H \times H$, $w(x,y) = (y, x)$
			$$ \Longrightarrow G\left( T^{-1} \right) = \big\{ \left(x, T^{-1} x\right) \colon x \in \DO{T} \big\} = w\left(G\left(T\right)\right) = \big\{ \left(Ty, y\right) \colon y \in \DO{T} \big\} $$
			\begin{align*}
				G\left( T^{-1} \right) & = G \left( \left( T^* \right)^{-1} \right) ~\overset{\hyperref[I.2:thm]{Proof}}{\underset{\hyperref[I.2:thm]{Thm. I.2}}{=}} w \left( V \left( G \left(T \right)^{\perp} \right) \right) \\
				& = V \left( w \left( G \left( T \right) \right)^{\perp} \right) =  V \left( w \left( G \left( T \right) \right) \right)^{\perp} \\
				& =  V \left(  G \left( T^{-1} \right) \right)^{\perp} \overset{\hyperref[I.2:thm]{Thm.}}{\underset{\ref{I.2:thm}}{=}} G \left( \left( T^{-1} \right)^* \right)
			\end{align*}
			$\Rightarrow T^{-1} = \left( T^{-1} \right)^*$
		\end{description}
	\end{proof}
\end{theorem}


\chapter{Representation Theorems}

\begin{theorem}[Riesz]
	Let $u \mapsto F(u)$ be a linear continuous function on $H$. Then $\exists! w \in H$:
	$$ F(u) = \langle u, w \rangle \quad \forall u \in H $$
\end{theorem}

\textbf{Lax-Milgram}: $V$ Hilbertspace, sesquilinear form is defined on $V \times V$, $(u, v) \mapsto \alpha (u, v)$ continuous with
$$ \left| \alpha (u, v) \right| \leq c \| u \| \|v \| \quad \forall u,v \in V $$
\textbf{Riesz}: $\exists$ linear map $A \colon V \rightarrow V$:
$$ \alpha(u, v ) = \langle A u, v \rangle $$

\index{coercive}
\begin{definition}
	A bilinear form $a \colon V \times V \rightarrow \R$ is \textbf{$V$-coercive} if there exists $\lambda > 0$ such that
	$$ a\left( u, u \right) \geq \lambda \| u \|^2 \quad \forall u \in V $$
\end{definition}

\begin{theorem}
	Let $a$ be a continuous sesquilinear and $V$-coercive on $V \times V$ then $A$ is an isomorphism.
	
	\begin{proof} ~\
		\begin{description}
			\item[~\hspace{0.25em}~Step 1:] $A$ is injective:
				\begin{align*}
					\| Au \| \| u \| \overset{C.S.}{\geq} \left| \langle A u, u \rangle \right| = \left| a(u, u) \right| \geq \lambda \| u \|^2 \tag*{$(+)$}
				\end{align*}  
				$\Rightarrow \| A u\| \geq \lambda \| u \|$ for all $u \in V$.
			\item[~\hspace{0.25em}~Step 2:] $A(V)$ is dense in $V$. Let $u \in V$ such that 
				$$ \langle A u, v \rangle = 0 \quad \forall v \in V $$
				take $v = u \Rightarrow a(u, u) = 0 \Rightarrow u = 0$. 
			\item[~\hspace{0.25em}~Step 3:] $\R(A) = A(V)$ is closed. Let $v_n$ be a sequence in $A(V)$ and let $u_n$ be such that
				$$ A u_n = v_n $$
				$\xRightarrow[]{(+)} u_n$ is a Cauchy sequence $\Rightarrow u_n \rightarrow u \in V$ und $A u_n \rightarrow A u \Rightarrow v_n \rightarrow A u \in A(V)$
			\item[~\hspace{0.25em}~Step 4:] $u = A^{-1} v \xRightarrow[]{(+)} \|A^{-1} v \| \leq \lambda^{-1} \| v \| ~\forall v \in V$. 
		\end{description}
	\end{proof}
\end{theorem}

Next we consider two Hilbert spaces $V, H$ with $V \subset H$ (the inclusion is continuous), i.e.
$$ \exists c < \infty \colon \quad \| u \|_H \leq c \| u \|_V \quad \forall u \in V $$
and we assume that $V$ is dense in $H$.

\begin{example}
	$V = W^{1,2}(\R^n)$, $H = L^2(\R^n)$
 		$$ \| u \|_L^2 \leq \| u \|_{W^{1,2}} $$	
 	There exists a natural injection from $H$ into $V'$. Let $h \in H$ then $V \ni u \mapsto \langle u, h \rangle_H$ is continuous on $V$ $\xRightarrow[\text{\hyperref[II.1:thm]{Thm. II.1}}]{} \exists l_h \in V'$:
 		$$ l_h(u) = \langle u, h \rangle_H \quad \forall u \in V $$
 	injectivity follows from density of $V$ in $H$. $V \subseteq H \subset V'$ cont. sesquilinear form $a$ on $V \times V$ which is $V$-coercive $\rightarrow$ Associate an unbounded operator $S$ with $a$
 		$$ \DO{S} \coloneqq \big\{ u \in V \colon a(u, v) \text{ is cont. on } V \text{ with respect to the topology induced by } H \big\} $$
\end{example}

% todo skipped a part! Unreadable

\begin{theorem} \label{thm:2.3}
	Let $a$ be a continuous sesquilinear form on $V$ which is $V$-coercive then $S$ is bijective from $\DO{S}$ into $H$ and $S^{-1} \in L(H, \DO{S})$. Moreover, $\DO{S}$ is dense in $H$.
	
	\begin{proof} ~\
		\begin{enumerate}[label=\arabic*\upshape)]
			\item $S$ injective: $\exists \alpha > 0$: 
				$$ \alpha \| u \|^2_H \leq C \alpha \| u \|^2_V \leq C \left| a (u,u) \right| = c \left| \langle S u, u \rangle_H \right| \leq c \| Su\|_H \| u \|_H, \quad \forall u \in \DO{S} $$
				$\Rightarrow \alpha \|u\|_H \leq c \| Su\|_H, ~ \forall u \in \DO{S} ~(+)$.
			\item $S$ surjective: ~\\
				Let $h \in H$. Choose $w \in V$ such that
				$$ \langle h, v \rangle_H = \overline{\langle v, h \rangle_H} = \overline{l_h(v)} = \langle w, v \rangle \quad \forall v \in V $$
				where we used Riesz' representation theorem in the last step. 
				$$\text{(Note: $l_h \in V' \Rightarrow \overline{l_h} \in$ continuous lineare form on $V$).} $$ % todo stimmt das hier? Schwer lesbar! 
				Define $u \coloneqq A^{-1} w \in V \Rightarrow a(u,v) = \langle A u, v \rangle_V = \langle w, v \rangle_V = \langle h, v \rangle_H$
				$$ \Rightarrow u \in \DO{S}, ~Su = h $$
				($V$ dense in $H$). $(+)$ implies that $S^{-1}$ is continuous.
			\item Density of $\DO{S}$: ~\\
				Let $h \in H$ such that $\langle u, h \rangle_H = 0 ~\forall u \in \DO{S}$. $S$ surjective $\exists v \in DO{S}$: $Sv = h$
				$$ \Rightarrow \langle Sv, u \rangle = 0 ~ \forall u \in \DO{S} $$
				$$ \Rightarrow \langle Sv, v \rangle_H = 0 \Rightarrow a(v, v) = 0 \Rightarrow v = 0 \Rightarrow h = 0 $$
				$a$ hermitian iff
				$$ a(u,v) = \overline{a(v, u)} \quad \forall u, v \in V $$
		\end{enumerate}
	\end{proof}	
\end{theorem}

\begin{theorem}
	Under the assumptions of \hyperref[thm:2.3]{Theorem II.3} and $a$ being hermitian it follows that
		\begin{enumerate}
			\item $S$ is closed
			\item $S = S^*$
			\item $\DO{S}$ dense in $V$
		\end{enumerate}
		
		\begin{proof} ~\
			\begin{enumerate}
				\item \hyperref[thm:1.4]{Theorem I.4}
				\item $a$ hermitian
					$$ \Rightarrow \langle Su, v \rangle_H = a(u, v) = \overline{a(v, u)} = \overline{\langle Sv, u \rangle_H} = \langle u, Sv \rangle_H \quad \forall u, v \in \DO{S} $$
					$\Rightarrow S$ symmetric $\Rightarrow \DO{S} \subset \DO{S^*}$. Let $v \in \DO{S^*}$, $S$ surjective 
						$$\Rightarrow v_0 \in \DO{S}: S v_0 = S^* v. $$ 
					For all $u \in \DO{S}$ we get
					$$ \langle Su, v_0 \rangle_H = \langle u, S v_0 \rangle_H = \langle u, S^* v \rangle_H = \langle S u, v \rangle_H $$
					$\Rightarrow v = v_0 \Rightarrow \DO{S} = \DO{S^*}$, $Sv = S^* v ~\forall v \in \DO{S}$.
				\item follows from \hyperref[thm:2.3]{Theorem II.3}
			\end{enumerate}	
		\end{proof}
\end{theorem}

\chapter{Friedrichs extension}

\begin{definition}
	Let $T_0$ be a symmetric unbounded operator with domain $\DO{T_0}$ we say that $T_0$ is semi bounded if $\exists c > 0$:
		$$ \langle T_0 u, u \rangle_H \geq - c \| u \|_H^2 \quad \forall u \in \DO{T_0} $$	
\end{definition}

\index{Hardy inequality}
\begin{example} ~\
	\begin{enumerate}
		\item Schrödinger Operator. $\R^m$, $H = L^2(\R^m)$, $\DO{T_0} = C_c^\infty(\R^m)$
				$$ T_0 \coloneqq - \Delta + V(x), $$
			$V \in C_0(\R^m)$ with $V(x) \geq - c ~\forall x \in \R^m$. For $u \in \DO{T_0}$
			$$ \langle T_0 u, u \rangle_H = \int_{\R^m} \left( \Delta u + V u \right) u = \underbrace{\int_{\R^m} \left| \nabla u \right|^2}_{\geq 0} + \underbrace{\int_{\R^m} V(x) \left| u(x) \right|^2}_{\geq - c \int |u|^2 = - c \|u \|^2_H} $$
		\item $S_z \coloneqq - \Delta - \frac{z}{r}$, whereas $r = |x|, z \in \R$ ~\\ 
		
			\textbf{Hardy inequality} in $\R^3 (m = 3)$:
				$$ \int_{\R^3} |x|^{-2} |u(x)|^2 dx \leq  4 \int_{\R^3} \left| \nabla u \right|^2 (x) dx \quad \forall u \in C_c^\infty(\R^m) $$
				\begin{proof}
					$\int_{\R^3} \left| \nabla u + \frac{1}{2} \frac{x}{|x|^2} u \right|^2 dx \geq 0$ 
					$$ \iff \int_{\R^3} \left| \nabla u \right|^2 + \frac{1}{4} \frac{|u|^2}{|x|^2} dx \geq - \int_{\R^3} \langle \nabla u(x), \frac{x}{|x|} \rangle u(x) dx $$
					now
					$$ -2 \int_{\R^3} \langle \nabla u, \frac {|x|^2}{|x|^2} \rangle u dx = - \int_{\R^3} \langle \nabla |u|^2, \frac{x}{|x|^2} dx = \int_{\R^3} |u|^2 \underbrace{\operatorname{div} \frac{x}{|x|^2}}_{= \frac{1}{|x|^2}} = \int_{\R^3} \frac{|u|^2}{|x|^2} dx $$ % todo hier stimmt doch was nicht
					$\Rightarrow \int |\nabla u|^2 \geq \frac{1}{4} \int \frac{|u|^2}{|x|^2}$. Now
					$$ \int_{\R^3} \frac{1}{r} |u(x)|^2 dx \leq \left( \int \frac{|u(x)|^2}{r^2} dx \right)^{\frac{1}{2}} \cdot \| u \|_L^2 $$
					$\int_{\R^3} \frac{1}{r^2} |u(x)|^2 dx \leq 4 \langle - \Delta u, u \rangle_L^2$
					$$ \Rightarrow \forall \epsilon > 0: \quad \int_{\R^3} \frac{1}{r} |u(x)|^2 dx \leq \epsilon \cdot \langle-\Delta u, u \rangle_L^2 + \frac{1}{\epsilon} \| u \|_{L^2}^2 $$
					hence 
					$$ \langle S_z u, u \rangle_{L^2} = \langle - \Delta u, u \rangle_{L^2} - z \langle \frac{u}{r}, u \rangle_{L^2} \geq (1 - \epsilon) \langle - \Delta u, u \rangle_{L^2} - \frac{z}{\epsilon} \| u \|^2_{L^2} $$
					Choose $\epsilon = \frac{1}{z} \Rightarrow \langle S_z u, u \rangle_{L^2} \geq - z^2 \| u\|^2_{L^2}$
				\end{proof}
	\end{enumerate}	
\end{example}

\index{Friedrichs extension}
\begin{theorem}
	A symmetric semibounded operator $T_0$ on $H$ with dense domain $\DO{T_0}$ admits a self-adjoint extension, called \textbf{Friedrichs extension}.
	
	\begin{proof}
		Replace $T_0$ by $T_0 + \lambda \mathbb{I}$ such that
		$$ \langle T_0 u, u \rangle_H \geq \|u\|_H^2 \quad \forall u \in \DO{T_0} $$
		$(u,v) \mapsto a_0(u,v) \coloneqq \langle T_0 u, v \rangle_H$ on $\DO{T_0} \times \DO{T_0}$
		$$ \Rightarrow a_0(u,u) \geq \| u \|^2_H $$
		Let $V$ be the completion in $H$ of $\DO{T_0}$ for the norm $u \mapsto \rho_0(u) = \sqrt{a_0(u,u)}$
		$\iff u \in H$ belongs to $V$ if $\exists u_n \in \DO{T_0}$ s.t. $u_n \rightarrow u$ in $H $ and $ (u_n)_{n \in \N}$ is a Cauchy sequence with respect to $\rho_0$ ~\\
		
		\enquote{Candidate Norm}:
		$$ \| u \|_V = \lim_{n \rightarrow \infty} \rho_0(u_n) $$
		where $u_n$ is as above.
	\end{proof}
\end{theorem}

\begin{lemma}
	Let $(x_n)_{n \in \N}$ be a Cauchy sequence
\end{lemma}

\appendix

\chapter*{Addendum}

\begin{theorem*}[Riesz' representation theorem, FA 17.2]
	To all $x' \in X'$ there exists a unique $x \in X$ such that
	$$ x'(y) = \langle y, x \rangle,  $$
	for $y \in X$ and $\|x'\|_{X'} = \|x\|_X$.
\end{theorem*}


% Exercises - End 

% Index									
\renewcommand{\indexname}{Stichwortverzeichnis}
\printindex


\end{document}