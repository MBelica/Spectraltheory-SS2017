\documentclass[12pt]{extreport} % Schriftgröße: 8pt, 9pt, 10pt, 11pt, 12pt, 14pt, 17pt oder 20pt

%% Packages
\usepackage{scrextend}
\usepackage{amssymb}
\usepackage{amsthm}
\usepackage{booktabs}
\usepackage{chngcntr}
\usepackage{cmap}
\usepackage{color}
\usepackage{csquotes}
\usepackage{enumitem}
\usepackage{float}
\usepackage{hyperref}
\usepackage{ulem}
\usepackage{lmodern}
\usepackage{makeidx}
\usepackage{mathtools}
\usepackage{xpatch}
\usepackage{pgfplots}
\pgfplotsset{compat=1.7}
\usetikzlibrary{calc}	
\usetikzlibrary{matrix}	

% Language Setup (English)
\usepackage[utf8]{inputenc} 
\usepackage[T1]{fontenc} 
\usepackage[english]{babel}

% Options
\makeatletter%%  
  % Linkfarbe, {0,0.35,0.35} für Türkis, {0,0,0} für Schwarz, {1,0,0} für Rot, {0,0,0.85} für Blau
  \definecolor{linkcolor}{rgb}{0,0.35,0.35}
  % Zeilenabstand für bessere Leserlichkeit
  \def\mystretch{1.2} 
  % Publisher definieren
  \newcommand\publishers[1]{\newcommand\@publishers{#1}} 
  % Enumerate im 1. Level: \alph für a), b), ...
  \renewcommand{\labelenumi}{\alph{enumi})} 
  % Enumerate im 2. Level: \roman für (i), (ii), ...
  \renewcommand{\labelenumii}{(\roman{enumii})}
  % Zeileneinrückung am Anfang des Absatzes
  \setlength{\parindent}{0pt} 
  % Für das Proof-Environment: 'Beweis:' anstatt 'Beweis.'
  \xpatchcmd{\proof}{\@addpunct{.}}{\@addpunct{:}}{}{} 
  % Nummerierung der Bilder, z.B.: Abbildung 4.1
  \@ifundefined{thechapter}{}{\def\thefigure{\thechapter.\arabic{figure}}} 
  % Chapter-Nummerierung beginnen bei (0):
  \setcounter{chapter}{0}
  % Chapter-Nummerierung
  \renewcommand\thechapter{\Roman{chapter}}
\makeatother%

% Meta Setup 
\title{Spectraltheory}
\author{Prof. Dr. Tobias Lamm}
\date{Sommersemester 2017}
\publishers{Karlsruher Institut für Technologie}

%% Math. Definitiones
\newcommand{\C}{\mathbb{C}}
\newcommand{\N}{\mathbb{N}}
\newcommand{\Q}{\mathbb{Q}}
\newcommand{\R}{\mathbb{R}}
\newcommand{\Z}{\mathbb{Z}}
\newcommand{\DO}[1]{\mathcal{D}\left( {#1} \right)}
\newcommand{\RO}[1]{\mathcal{R}\left( {#1} \right)}

\newtheoremstyle{named}{}{}{\normalfont}{}{\bfseries}{:}{0.25em}{#2 \thmnote{#3}}
\newtheoremstyle{nnamed}{}{}{\normalfont}{}{\bfseries}{:}{0.25em}{\thmnote{#3}}
\newtheoremstyle{itshape}{}{}{\itshape}{}{\bfseries}{:}{ }{}
\newtheoremstyle{normal}{}{}{\normalfont}{}{\bfseries}{:}{ }{}
\renewcommand*{\qed}{\hfill\ensuremath{\square}}

\theoremstyle{named}
\newtheorem{unnamedtheorem}{Theorem} \counterwithin{unnamedtheorem}{chapter}
\theoremstyle{nnamed}
\newtheorem*{unnamedtheorem*}{Theorem} 

\theoremstyle{itshape}
\newtheorem{definition}[unnamedtheorem]{Definition}

\theoremstyle{normal}
\newtheorem*{recall}{Recall}
\newtheorem*{example}{Example}

%% Template
\makeatletter%
\DeclareUnicodeCharacter{00A0}{ } \pgfplotsset{compat=1.7} \hypersetup{colorlinks,breaklinks, urlcolor=linkcolor, linkcolor=linkcolor, pdftitle=\@title, pdfauthor=\@author, pdfsubject=\@title, pdfcreator=\@publishers}\DeclareOption*{\PassOptionsToClass{\CurrentOption}{report}} \ProcessOptions \def\baselinestretch{\mystretch} \setlength{\oddsidemargin}{0.125in} \setlength{\evensidemargin}{0.125in} \setlength{\topmargin}{0.5in} \setlength{\textwidth}{6.25in} \setlength{\textheight}{8in} \addtolength{\topmargin}{-\headheight} \addtolength{\topmargin}{-\headsep} \def\pulldownheader{ \addtolength{\topmargin}{\headheight} \addtolength{\topmargin}{\headsep} \addtolength{\textheight}{-\headheight} \addtolength{\textheight}{-\headsep} } \def\pullupfooter{ \addtolength{\textheight}{-\footskip} } \def\ps@headings{\let\@mkboth\markboth \def\@oddfoot{} \def\@evenfoot{} \def\@oddhead{\hbox {}\sl \rightmark \hfil \rm\thepage} \def\chaptermark##1{\markright {\uppercase{\ifnum \c@secnumdepth >\m@ne \@chapapp\ \thechapter. \ \fi ##1}}} \pulldownheader } \def\ps@myheadings{\let\@mkboth\@gobbletwo \def\@oddfoot{} \def\@evenfoot{} \def\sectionmark##1{} \def\subsectionmark##1{}  \def\@evenhead{\rm \thepage\hfil\sl\leftmark\hbox {}} \def\@oddhead{\hbox{}\sl\rightmark \hfil \rm\thepage} \pulldownheader }	\def\chapter{\cleardoublepage  \thispagestyle{plain} \global\@topnum\z@ \@afterindentfalse \secdef\@chapter\@schapter} \def\@makeschapterhead#1{ {\parindent \z@ \raggedright \normalfont \interlinepenalty\@M \Huge \bfseries  #1\par\nobreak \vskip 40\p@ }} \newcommand{\indexsection}{chapter} \patchcmd{\@makechapterhead}{\vspace*{50\p@}}{}{}{}\def\Xint#1{\mathchoice
    {\XXint\displaystyle\textstyle{#1}} {\XXint\textstyle\scriptstyle{#1}} {\XXint\scriptstyle\scriptscriptstyle{#1}} {\XXint\scriptscriptstyle\scriptscriptstyle{#1}} \!\int} \def\XXint#1#2#3{{\setbox0=\hbox{$#1{#2#3}{\int}$} \vcenter{\hbox{$#2#3$}}\kern-.5\wd0}} \def\dashint{\Xint-} \def\Yint#1{\mathchoice {\YYint\displaystyle\textstyle{#1}} {\YYYint\textstyle\scriptscriptstyle{#1}} {}{} \!\int} \def\YYint#1#2#3{{\setbox0=\hbox{$#1{#2#3}{\int}$} \lower1ex\hbox{$#2#3$}\kern-.46\wd0}} \def\YYYint#1#2#3{{\setbox0=\hbox{$#1{#2#3}{\int}$}  \lower0.35ex\hbox{$#2#3$}\kern-.48\wd0}} \def\lowdashint{\Yint-} \def\Zint#1{\mathchoice {\ZZint\displaystyle\textstyle{#1}}{\ZZZint\textstyle\scriptscriptstyle{#1}} {}{} \!\int} \def\ZZint#1#2#3{{\setbox0=\hbox{$#1{#2#3}{\int}$}\raise1.15ex\hbox{$#2#3$}\kern-.57\wd0}} \def\ZZZint#1#2#3{{\setbox0=\hbox{$#1{#2#3}{\int}$} \raise0.85ex\hbox{$#2#3$}\kern-.53\wd0}} \def\highdashint{\Zint-} \DeclareRobustCommand*{\onlyattoc}[1]{} \newcommand*{\activateonlyattoc}{ \DeclareRobustCommand*{\onlyattoc}[1]{##1} } \AtBeginDocument{\addtocontents{toc} {\protect\activateonlyattoc}} \newcommand{\RightArrow}{\xRightarrow[]{ ~ ~ }} \newcommand{\LeftArrow}{\xLeftarrow[]{ ~ ~ }} \newcommand{\rightArrow}{\xrightarrow[]{ ~ ~ }} \newcommand{\leftArrow}{\xleftarrow[]{ ~ ~ }}
	% Titlepage
	\def\maketitle{ \begin{titlepage} 
			~\vspace{3cm} 
		\begin{center} {\Huge \@title} \end{center} 
	 		\vspace*{1cm} 
	 	\begin{center} {\large \@author} \end{center} 
	 	\vspace*{-0.5cm}
	 	\begin{center} \@date \end{center} 
	 		\vspace*{7cm} 
	 	\begin{center} \@publishers \end{center} 
	 		\vfill 
	\end{titlepage} }
\makeatother%

% Create Index
\makeindex 

\begin{document}

\pagenumbering{Alph}
\begin{titlepage}
	\maketitle
	\thispagestyle{empty}
\end{titlepage}

% Table of Contents
\tableofcontents
\thispagestyle{empty}

% Lecture Notes - Start 			
\pagenumbering{arabic}

\chapter{Unbounded operators, adjoint and self-adjoint operators}

Let $H$ be a separable Hilbert space, $\langle \cdot, \cdot \rangle$ denote the scalar product on $H$. ~\bigskip

A linear operator $T$ in $H$ is a linear map $u \mapsto Tu$ defined on a subspace $\DO{T}$ of $H$, and we call $\DO{T}$ the domain of $T$. ~\smallskip

For $T \colon \DO{T} \rightarrow H$ we denote the range of $T$ with
$$ \RO{T} \coloneqq \operatorname{Image}\left( T \right). $$

\index{bounded}
We say that $T$ is \textbf{bounded} if it is continuous from $\DO{T}$ into $H$, with respect to the topology induced by $H$. ~\smallskip

If $\DO{T} = H$ we recall the definition of bounded operators from the \href{https://github.com/MBelica/Funktionalanalysis-WS2015}{functional analysis} course. ~\bigskip

\index{dense}
From now on, if $\DO{T} \neq H$ we will assume that $\DO{T}$ is \textbf{dense} in $H$, i.e. $\overline{\DO{T}} = H$. In this case, if $T$ is bounded then $T$ has a unique continuous extension to all of $H$.
$$ \RightArrow T \text{ bounded is ...} $$ % todo Was steht hier?

\index{closed}
\begin{recall}
	An operator is called \textbf{closed} if the graph 
	$$ G(T) \coloneqq \left\{ (x, y) \in H \times H \colon x \in \DO{T}, y = Tx \right\} $$
	is closed in $H \times H$.
\end{recall}

\newpage

\begin{definition} \label{i.1:def}
	Let $T \colon \DO{T} \rightarrow H$ be a (linear) operator with $\DO{T}$ dense in $H$. Then $T$ is called \textbf{closed} if the conditions
	
	$$ 
		\begin{rcases}
			u_n \in \DO{T} \\
			u_n \rightarrow u \text{ in } H \\
			T u_n \rightarrow v \text{ in } H
		\end{rcases} \RightArrow u \in \DO{T}, v = Tu
	$$
	hold.
\end{definition}

\begin{example} ~\
	\begin{enumerate}
		\item Let $T_0 = - \Delta$, $H = \ell^2(\R^n)$ and $\DO{T_{0}} = C_c^\infty(\R^n)$ dense in $H$. ~\\
			Take $u \in W^{2,2}(\R^n) \setminus C_c^{\infty}(\R^n)$
			$$ \xRightarrow[]{\text{densly}} \exists (u_n)_{n \in \N} \in C_c^{\infty}(\R^n) \colon ~ u_n \rightarrow u \text{ in } W^{2,2}(\R^n) $$
			$\left( u_n, -\Delta u_n \right) \in G(T_0)$ converges in $\ell^2 \times L^2$ to $(u, -\Delta u) \notin G(T_0)$.
		\item Let $T_1 = - \Delta$, $\DO{T_1} = W^{2,2}(\R^n)$ and $H = L^2(\R^n)$. For $u_n \in \DO{T_1}$ with 
			$$ u_n \rightarrow u \text{ in } H \text{ and } \left( - \Delta u_n \right) \rightarrow u \text{ in } L^2 $$
			follows that $- \Delta u = v \in L^2(\R^n)$ weakly, i.e. $\forall \varphi \in C_c^{\infty}(\R^n)$:
			$$ \int_{\R^n} v \varphi \leftArrow \int_{\R^n} \left( - \Delta u_n \right) \varphi = \int_{\R^n} u_n \left( - \Delta \varphi \right) \rightArrow \int_{\R^n} u \left( - \Delta \varphi \right). $$
	\end{enumerate}
\end{example}


% Lecture Notes - End 			
\appendix 

% Index									
\renewcommand{\indexname}{Stichwortverzeichnis}
\printindex


\end{document}